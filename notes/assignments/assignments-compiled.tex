\documentclass[12pt, oneside]{article}

\usepackage[letterpaper, scale=0.8, centering]{geometry}
\usepackage{fancyhdr}
\setlength{\parindent}{0em}
\setlength{\parskip}{1em}

\pagestyle{fancy}
\fancyhf{}
\renewcommand{\headrulewidth}{0pt}
\rfoot{{\footnotesize Copyright Mia Minnes, 2021, Version \today~(\thepage)}}

\author{CSEXF22}

\newcommand{\instructions}{{\bf For all HW assignments:}

Weekly homework may be done individually or in groups of up to 3 students. 
You may switch HW partners for different HW assignments. 
The lowest HW score will not be included in your overall HW average. 
Please ensure your name(s) and PID(s) are clearly visible on the first page of your homework submission.

All submitted homework for this class must be typed. 
Diagrams may be hand-drawn and scanned and included in the typed document. 
You can use a word processing editor if you like (Microsoft Word, Open Office, Notepad, Vim, Google Docs, etc.) 
but you might find it useful to take this opportunity to learn LaTeX. 
LaTeX is a markup language used widely in computer science and mathematics. 
The homework assignments are typed using LaTeX and you can use the source files 
as templates for typesetting your solutions\footnote{To use this template, copy the source file (extension \texttt{.tex}) 
to your working directory or upload to Overleaf.}.


{\bf Integrity reminders}
\begin{itemize}
\item Problems should be solved together, not divided up between the partners. The homework is
designed to give you practice with the main concepts and techniques of the course, 
while getting to know and learn from your classmates.
\item You may not collaborate on homework with anyone other than your group members.
You may ask questions about the homework in office hours (of the instructor, TAs, and/or tutors) and 
on Piazza (as private notes viewable only to the Instructors).  
You \emph{cannot} use any online resources about the course content other than the class material 
from this quarter -- this is primarily to ensure that we all use consistent notation and
definitions we will use this quarter.
\item Do not share written solutions or partial solutions for homework with 
other students in the class who are not in your group. Doing so would dilute their learning 
experience and detract from their success in the class.
\end{itemize}

}
\input{../../resources/discrete-math-packages}



\title{HW1 Definitions and Notation}
\date{Due: Tuesday, October 5, 2021 at 11:00PM on Gradescope}

\begin{document}
\maketitle
\thispagestyle{fancy}


{\bf In this assignment,}

You will practice reading and
applying definitions to get comfortable working with mathematical language. As
a result, you can expect to spend more time reading the questions and looking
up notation than doing calculations.

\instructions

You will submit this assignment via Gradescope
(\href{https://www.gradescope.com}{https://www.gradescope.com}) 
in the assignment called ``hw1-definitions-and-notation''.


{\bf Resources}: To review the topics you are working with 
for this assignment, see the class material from  Week 0 and 1.
We will post frequently asked questions and our answers to them in a 
pinned Piazza post.

{\bf Assigned questions}

\begin{enumerate}

\item ({\it Graded for correctness}\footnote{This means your solution will be
evaluated not only on the correctness of your answers, but on your ability to 
present your ideas clearly and logically. You should explain how you arrived at 
your conclusions, using 
mathematically sound reasoning. Whether you use formal proof techniques or 
write a more informal argument for why 
something is true, your answers should always be well-supported. Your goal 
should be to convince the reader that 
your results and methods are sound.}) Each of the sets below is described 
using set builder notation or as a result of set operations
applied to other known sets.  Rewrite them using the roster method.

Remember our discussions of data-types: use clear notation that 
is consistent with our class notes and definitions 
to communicate the data-types of the elements in each set.


\rule{0.5\textwidth}{.4pt}

{\it Sample response that can be used as reference for the detail expected 
in your answer:} 

The set $\{ \A \} \circ \{ \A\U, \A\C, \A\G\}$ can be written using
the roster method as 
\[
\{ \A\A\U, \A\A\C, \A\A\G \}
\]
because set-wise concatenation gives a set whose elements are 
all possible results of concatenating an element of the 
left set with an 
element of the right set. Since the left set in this example only
has one element ($\A$), each of the elements of the set we 
described starts with $\A$. There are three elements of this set, 
one for each of the distinct elements of the right set.

\rule{0.5\textwidth}{.4pt}

\begin{enumerate}
\item $$\{ x \in S \mid rnalen(x) = 1 \} \circ \{x \in S \mid rnalen(x) = 1 \}$$
where $S$ is the set of RNA strands and $rnalen$ is the recursively defined
function that we discussed in class.
\item $$\{ (r,g,b) \in C \mid r+g+b = 1\}$$ where 
$C = \{ (r,g,b) \mid 0 \leq r \leq 255, 0 \leq g \leq 255, 0 \leq b \leq 255, r \in \mathbb{N}, g \in \mathbb{N}, b \in \mathbb{N} \}$
is the set that you worked with in Monday's review quiz.
\item $$\{ a \in \mathbb{Z} \mid  a \textbf{ div } 2 = a \textbf{ mod } 2\}$$
\end{enumerate}

\item ({\it Graded for fair effort completeness}\footnote{This means 
you will get full credit so long as your submission demonstrates honest 
effort to answer the question. You will not be penalized for incorrect answers.}) 

\begin{enumerate}
    \item In Wednesday's review quiz, you considered some attempted 
    recursive definitions for the function
    with domain $\mathbb{N}$ and with codomain $\mathbb{Z}$
    which gives $2^n$ for each $n$. 
    Write out a correct recursive definition of this function.
    \item How would your answer to part (a) change if we consider
    a new function with the same domain and rule but whose codomain 
    is $\mathbb{R}$?
    \item How would your answer to part (a) change if we consider
    a new function with the same codomain and rule but whose domain 
    is $\mathbb{R}$?
    \item Write a recursive definition of the function with domain $\mathbb{Z}^+$,
    codomain $\mathbb{Z}^+$ and which gives $n!$ for each $n$. The $!$ symbol
    is the ``factorial'' symbol and means that we need to multiple $n$ by each of the integers
    between it and $1$ inclusive. For example, $5! = 5 \cdot 4 \cdot 3 \cdot 2 \cdot 1 = 120$.
\end{enumerate}

\item ({\it Graded for correctness}) Recall the function
$d_0$ which takes an ordered pair of ratings $3$-tuples and returns a measure
of the distance between them 
given by
\[
d_0 (~(~ (x_1, x_2, x_3), (y_1, y_2, y_3) ~) ~) = \sqrt{ (x_1 - y_1)^2 + (x_2 - y_2)^2 + (x_3 -y_3)^2}
\]

\rule{0.5\textwidth}{.4pt}

{\it Sample response that can be used as reference for the detail expected 
in your answers for this question: } 

To give an example of two $3$-tuples that are $d_0$ distance $1$ from each other, 
consider the $3$-tuples $(1,0,0)$ and $(0,0,0)$. We calculate the function application:
\begin{align*}
    d_0 (~(~ (1, 0,0), (0,0,0) ~) ~) &= \sqrt{ (1 - 0)^2 + (0 - 0)^2 + (0 -0)^2} = \sqrt{1^2 + 0^2 + 0^2} = \sqrt{1} = 1,
\end{align*}
which is the result required for this example.

\rule{0.5\textwidth}{.4pt}

\begin{enumerate}
    \item Give an example of three $3$-tuples 
    \begin{align*}
    &(x_{1,1}, x_{1,2}, x_{1,3}) \\
    &(x_{2,1}, x_{2,2}, x_{2,3}) \\
    &(x_{3,1}, x_{3,2}, x_{3,3}) \\
    \end{align*}
    that are all $d_0$ distance {\bf greater than} $1$ from each other.  
    In other words, for each $i$ and $j$ between $1$ and 
    $3$ (with $i \neq j$), 
    \[
    d_{0}(~(~(x_{i,1}, x_{i,2}, x_{i,3})  , (x_{j,1}, x_{j,2}, x_{j,3})~)~) > 1
    \]
    
    Your answer should include  {\bf both} specific values for each example $3$-tuple {\bf and} a justification 
    of your examples with (clear, correct, complete) calculations and/or references to definitions and connecting them with
    the desired conclusion.\\
    
    {\it To think about}: 
    how will you justify that the square root of a number is greater than $1$?
    Are calculations from a calculator accurate enough to help us?

    \item What is the range of values that results from applying the function $d_0$
    to ordered pairs of $3$-tuple ratings? That is, what are the smallest and largest
    possible results?

    Your answer should include  {\bf both} specific values for the smallest and largest 
    possible results {\bf and} a justification 
    of your answers with (clear, correct, complete) calculations and/or references to definitions and connecting them with
    the desired conclusion.
\end{enumerate}
\end{enumerate}
\newpage

\title{HW2 Numbers}
\date{Due: Tuesday, October 12, 2021 at 11:00PM on Gradescope}


\maketitle
\thispagestyle{fancy}


{\bf In this assignment,}

You will consider multiple number representations and how they connect 
to applications in computer science. You will also practice tracing
and working with algorithms.

Instructions and academic integrity reminders for all homework assignments in 
CSE20 this quarter are on the class website and on the hw1-definitions-and-notations
assignment.


You will submit this assignment via Gradescope
(\href{https://www.gradescope.com}{https://www.gradescope.com}) 
in the assignment called ``hw2-numbers''.


{\bf Resources}: To review the topics you are working with 
for this assignment, see the class material from Friday of Week 1 and Monday
 and Wednesday of Week 2.
We will post frequently asked questions and our answers to them in a 
pinned Piazza post.

{\bf Assigned questions}

\begin{enumerate}
    \item ({\it Graded for fair effort completeness}\footnote{This means 
    you will get full credit so long as your submission demonstrates honest 
    effort to answer the question. You will not be penalized for incorrect answers.})
    Pick a nonnegative integer between 100 and 1000 (inclusive) and express it using at 
    least three different representations, 
    at least two of which must be ones discussed in class. Include at least one
    trace of using procedure $baseb1$ from page 16 of the Week 0 and 1 notes
    to calculate the base $b_1$ expansion of your
    number (for your choice of $b_1$) and include at least one trace of using 
    procedure $baseb2$ from page 16 of the Week 0 and 1 notes
    to calculate the base $b_2$ expansion of your number
    (for your choice of $b_2$). 
    
    {\it Extra; not for credit:} Consider choosing representations that might be useful in 
    some way; what would make one representation more useful than another?

    \newpage
    \item In this question, we will use recursive definitions to give a precise description 
    of the set of strings that form octal (base $8$) expansions and the values
    of those expansions. Let's start by defining the set of possible coefficients
    \[
        C_8 = \{ 0, 1, 2, 3, 4, 5, 6, 7\}
    \]

    \begin{enumerate}
    \item ({\it Graded for fair effort completeness}) Fill in the recursive definition of the 
    set of all strings that form octal expansions, $S_8$: 
    
    {\bf Definition} The set $S_8$ is defined (recursively) by:
    \[
    \begin{array}{ll}
    \textrm{Basis Step: } & \textrm{If } x \in \underline{\phantom{\hspace{2in}}} \textrm{, then } x \in S_8\\
    \textrm{Recursive Step: } & \textrm{If } s \in S_8 \textrm{ and } x \in C_8 \textrm{, then }
    \underline{\phantom{\hspace{2in}}} \in S_8
    \end{array}
    \]
    
    \item  ({\it Graded for correctness}\footnote{This means your solution will be
    evaluated not only on the correctness of your answers, but on your ability to 
    present your ideas clearly and logically. You should explain how you arrived at 
    your conclusions, using 
    mathematically sound reasoning. Whether you use formal proof techniques or 
    write a more informal argument for why 
    something is true, your answers should always be well-supported. Your goal 
    should be to convince the reader that 
    your results and methods are sound.})
    Consider the function $v_8: S_8 \to \mathbb{Z}^+$ defined recursively by 
    
    \begin{quote}
    Basis Step: If $x \in C_8$, then $v_8 (x) = x$.
    
    Recursive Step: If $s \in S_8$ and $x \in C_8$, then $v_8 (sx) = 8v_8(s) + x$, where 
    the input $sx$ is the result of string concatenation and the output $8 v_8 (s)$
    is the result of integer multiplication.
    \end{quote}

    Calculate $v_8(104)$, including all steps in your calculation and justifications for them.

    \item ({\it Graded for fair effort completeness})
    It turns out\footnote{We'll be able to prove this in Week 7 or so, once
    we've talked about induction.} that for any string $u$ in $S_8$, the value of the 
    octal expansion $(u)_8$ equals $v_8(u)$. Using this fact, write an expression 
    relating the value of $(u000)_8$ to the value of $(u)_8$ and justify it. 
    \end{enumerate}
    
    
    \item ({\it Graded for correctness}) 
    Recall that, mathematically, a color can be represented as a 
    $3$-tuple $(r, g, b)$ where $r$
    represents the red component, $g$ the green component, $b$ the blue component 
    and where each of $r$, $g$, $b$ must be from the collection 
    $\{x \in \mathbb{N}\mid 0 \leq x \leq 255 \}$.
    
    As an {\bf alternative} representation, in this assignment
    we'll use base $16$ fixed-width expansions to represent colors
    as single numbers.
    
    {\bf Definition}: A {\bf hex color} is a nonnegative
    integer less than or equal to $16777215$. 
    For $n$ a hex color, we define its red, green, and blue components
    by first writing its base $16$ fixed-width $6$ expansion
    $$n = (r_1r_2g_1g_2b_1b_2)_{16,6}$$ and 
    then defining
    $(r_1r_2)_{16,2}$ is the red
    component, $(g_1g_2)_{16,2}$ is the green component, and $(b_1b_2)_{16,2}$ is the
    blue component.
    
    \newpage
    \rule{0.5\textwidth}{.4pt}

    {\it Sample response that can be used as reference for the detail expected 
    in your answer:} 
    
    In RGB codes\footnote{You can use online tools to visualize the colors associated
    with different values for the red, green, and blue components, 
    e.g. \url{https://www.w3schools.com/colors/colors_rgb.asp}. }
    white is represented as maximum red, maximum green, and maximum blue 
    and so has
    $(FF)_{16,2}$ as each of these components. This means that the hex color for white
    is $(FFFFFF)_{16,6}$ which is the value 
    \[
        15\cdot 16^5 + 15 \cdot 16^4 + 15 \cdot 16^3 + 15 \cdot 16^2 + 15 \cdot 16^1 + 15 \cdot 16^0 
        = 16^6 - 1 = 16777215
    \]
    \rule{0.5\textwidth}{.4pt}

    \begin{enumerate}
    \item  Write the hex color representing red (with no green or blue) 
    in base $16$ fixed-width $6$ and also calculate its value 
    (using usual mathematical conventions).
    Include your (clear, correct, complete) calculations.
    \item  Write the hex color representing green (with no red or blue) 
    in base $16$ fixed-width $6$ and also calculate its value
    (using usual mathematical conventions).
    Include your (clear, correct, complete) calculations.
    \item  Write the hex color representing blue (with no red or green) 
    in base $16$ fixed-width $6$ and also calculate its values
    (using usual mathematical conventions).
    Include your (clear, correct, complete) calculations.
    \item The human eye can't distinguish between some hex colors because of 
    physical limitations. Give an example of two hex colors $c_1$ and $c_2$ such that
    they look indistinguishable (the colors they represent are very very similar) but
        \[
            c_1 - c_2 > 50000
        \]
    Justify your choice 
    with (clear, correct, complete) calculations and/or references to definitions, 
    and connecting these
    calculations and/or definitions with
    the desired properties.  Include squares with each of your two colors so 
    that we can see how indistinguishable they are. 

    Pro tip: To show a color, you can use the following LaTeX source code:
    
        \verb|\definecolor{UCSDaccent}{RGB}{0,198,215}|

        \verb|\textcolor{UCSDaccent}{\rule{1cm}{1cm}}|

    which produces 
    \definecolor{UCSDaccent}{RGB}{0,198,215}
    \textcolor{UCSDaccent}{\rule{1cm}{1cm}}

    Notice that the code to define the color uses the decimal-like values 
    for each of the red, green, and blue components. For the UCSD accent color we defined, 
    the base $16$ fixed-width $2$ values are: red is $(00)_{16,2}$, green is $(C6)_{16,2}$,
    blue is $(D7)_{16,2}$.

    {\it Extra; not for credit:} What does this mean about the choice of hex color for
    representing colors? What are advantages and disadvantages of this representation?

    \end{enumerate}

    \item ({\it Graded for correctness}) In class (Week 2 notes page 7), we discussed fixed-width addition. In this
    question we will look at fixed-width multiplication. The algorithm for fixed-width 
    multiplication is to multiply using the usual long-multiplication algorithm 
    (column-by-column and carry), and dropping all leftmost columns so the result is the same 
    width as the input terms. For each of the examples below, consider whether 
    this algorithm gives the correct value for the product of the two numbers, based on
    the way the bitstrings are interpreted.

    \rule{0.5\textwidth}{.4pt}

    {\it Sample response that can be used as reference for the detail expected 
    in your answer:} 
    
    The fixed-width $5$ multiplication of $[00101]_{2c,5}$ and $[00101]_{2c,5}$ 
    does not give the correct
    value for the product, as we can see from the following calculation.
    
    First, we calculate the values: 
    \[
        [00101]_{2c,5} = 0\cdot (-2^4) + 0\cdot 2^3 + 1 \cdot 2^2 + 0\cdot 2^1 + 1 \cdot 2^0 = 4 + 1 = 5
    \]
    so the correct value for the product is $5 \cdot 5 = 25$, which cannot be represented
    in 2s complement width 5 (the largest positive number that can be represented in 
    2s complement width 5 is $[01111]_{2c,5} = 8 + 4 + 2 + 1 = 15$).
    
    When we perform the fixed-width $5$ multiplication algorithm:
    \begin{align*}
            & 0~ 0~ 1~ 0~ 1\\
     \times & 0~ 0~ 1~ 0~ 1\\
     &\overline{0~ 0~ 1~ 0~ 1}\\
     + {0~} & {0~0~0~0~~}\\
     + {0~0~} & {1~0~1~~}\\
     \overline{\phantom{0~0~}}&\overline{1~ 1~ 0~ 0~ 1}\\
    \end{align*}

    we get $[11001]_{2c,5} = 1\cdot (-2^4) + 1 \cdot 2^3 + 0 \cdot 2^2 + 0 \cdot 2^1 + 1 \cdot 2^0 = -16 + 8 + 1 = -7$, 
    which is not the required value.

    \rule{0.5\textwidth}{.4pt}


    \begin{enumerate}
        \item Does the fixed-width $5$ multiplication of $[11101]_{2c,5}$ and 
        $[11011]_{2c,5}$ give the correct value for the product?
        Justify your answer  
        with (clear, correct, complete) calculations and/or references to definitions, 
        and connecting these
        calculations and/or definitions with
        your answer.
        
        \item Does the fixed-width $5$ multiplication of $[00100]_{2c,5}$ and 
        $[11100]_{2c,5}$ give the correct value for the product?
        Justify your answer  
        with (clear, correct, complete) calculations and/or references to definitions, 
        and connecting these
        calculations and/or definitions with
        your answer.
    \end{enumerate}
    
\end{enumerate}
\newpage

\title{HW3 Circuits and Logic}
\date{Due: Tuesday, October 26, 2021 at 11:00PM on Gradescope}


\maketitle
\thispagestyle{fancy}

{\bf In this assignment,}

You will consider how circuits and logic can be used to represent
mathematical and technical claims. You will use propositional 
and predicate logic to evaluate these claims.

Instructions and academic integrity reminders for all homework assignments in 
CSE20 this quarter are on the class website and on the hw1-definitions-and-notations
assignment.

You will submit this assignment via Gradescope
(\href{https://www.gradescope.com}{https://www.gradescope.com}) 
in the assignment called ``hw3-circuits-and-logic''.

{\bf Resources}: To review the topics you are working with 
for this assignment, see the class material from Week 3 and Week 4.
We will post frequently asked questions and our answers to them in a 
pinned Piazza post.

{\bf Assigned questions}

\begin{enumerate}

   \item Imagine a friend suggests the following argument to you: ``The compound proposition
   \[
   (x \to y) \to z
   \]
   is logically equivalent to 
   \[
   x \to (y \to z)
   \]
   because I can transform one to the other using the following sequence of logical equivalences
   like distributivity and associativity and using that $(p \to q) \equiv (\lnot p \lor q)$: 
   \[
      (x \to y) \to z \equiv
      \lnot (\lnot x \lor y) \lor z \equiv
      \lnot x \lor (\lnot y \lor z) \equiv
      x \to (\lnot y \lor z) \equiv x \to (y \to z)
   \]
   so conditionals
   are associative just like conjunctions and disjunctions".
   
   \begin{enumerate}
   \item ({\it Graded for correctness}\footnote{This means your solution will be
   evaluated not only on the correctness of your answers, but on your ability to 
   present your ideas clearly and logically. You should explain how you arrived at your conclusions, using 
   mathematically sound reasoning. Whether you use formal proof techniques or write a more informal argument for why 
   something is true, your answers should always be well-supported. Your goal should be to convince the reader that 
   your results and methods are sound.}) Prove to your friend that they made a mistake by giving a truth
   assignment to the propositional variables where 
   the two compound propositions 
   $ (x \to y) \to z$ and $ x \to (y \to z)$ have different truth values.
   Justify your choice by evaluating these compound propositions using the definitions of the logical connectives 
   and include enough intermediate steps so that a student in CSE 20 who may be 
   struggling with the material can still follow along with your reasoning.
   
   \item ({\it Graded for fair effort completeness}\footnote{This means you will get full credit so long as your submission 
   demonstrates honest effort to answer the question. You will not be penalized for incorrect answers.}) 
   Help your friend find the problem in their argument by pointing out which step(s) were incorrect.
   
   \item ({\it Graded for fair effort completeness}) Give {\bf three} different compound propositions
   that are actually logically equivalent to (and not the same as)
   \[
   x \to (y \to z)
   \]
   Justify each one of these logical equivalences either by applying a sequence of logical equivalences
   or using a truth table.  Notice that you can use other logical operators (e.g. $\lnot, \lor, \land, \oplus, \to, 
   \leftrightarrow$) 
   when constructing your compound propositions.

   {\it Bonus; not for credit (do not hand in)}: How would you translate each of the equivalent compound
   propositions in English? Does doing so help illustrate why they are equivalent?
   \end{enumerate}
   
   \item For each part of this question you will use the following input-output definition table 
   with four inputs $x_3$, $x_2$, $x_1$, $x_0$
   
   \begin{center}
   \begin{tabular}{cccc|c}
   $x_3$ & $x_2$ & $x_1$ & $x_0$ & $out$\\
   \hline
   $1$ & $1$ & $1$ & $1$ & $1$\\
   $1$ & $1$ & $1$ & $0$ & $0$\\
   $1$ & $1$ & $0$ & $1$ & $0$\\
   $1$ & $1$ & $0$ & $0$ & $0$\\
   $1$ & $0$ & $1$ & $1$ & $0$\\
   $1$ & $0$ & $1$ & $0$ & $1$\\
   $1$ & $0$ & $0$ & $1$ & $0$\\
   $1$ & $0$ & $0$ & $0$ & $0$\\
   $0$ & $1$ & $1$ & $1$ & $0$\\
   $0$ & $1$ & $1$ & $0$ & $0$\\
   $0$ & $1$ & $0$ & $1$ & $1$\\
   $0$ & $1$ & $0$ & $0$ & $0$\\
   $0$ & $0$ & $1$ & $1$ & $0$\\
   $0$ & $0$ & $1$ & $0$ & $0$\\
   $0$ & $0$ & $0$ & $1$ & $0$\\
   $0$ & $0$ & $0$ & $0$ & $1$\\
   \end{tabular}
   \end{center}
   \begin{enumerate}   
   \item  ({\it Graded for fair effort completeness}) 
   Construct a compound proposition that implements this input-output table
   and draw a combinatorial circuit corresponding to this compound proposition. We recommend
   the following steps:
   
   \begin{itemize}
   \item Draw symbols for the inputs on the left-hand-side and for the output on the right-hand side.
   \item Construct an expression for $out$ using (some of) the inputs 
   $x_3, x_2, x_1, x_0$ and the logic gates XOR, AND, OR, NOT. {\it Hint:} are normal forms helpful here?
   How do you choose which normal form to use?
   \item Draw and label the gates corresponding to the expression you construct, and connect appropriately with wires.
   \end{itemize}
   
   \item ({\it Graded for correctness}) 
   For each of the following predicates, consider whether this input-output table (and the
   logic circuit from part (a)) implements the rule for 
   the predicate.
   If yes, explain why using the definitions of the operations involved and consider {\bf all} possible 
   inputs. If no, explain why not by providing specific example input values 
   and using the input-output definition table to compute the logic circuit's output for this example input
   and then comparing with the value of the predicate in question to justify your example.

   \begin{enumerate}
      \item $P_1: \{0,1\}\times \{0,1\} \times \{0,1\} \times \{0,1\} \to \{T, F\}$ given by 
      $$P_1(~(x_3,x_2,x_1,x_0)~) = \begin{cases}  T \quad &\text{when~}(x_3x_2x_1x_0)_{2,4} \leq 5 \\ F &\text{otherwise}\end{cases}$$
      \item $P_2: \{0,1\}\times \{0,1\} \times \{0,1\} \times \{0,1\} \to \{T, F\}$ given by 
      $$P_2(~(x_3,x_2,x_1,x_0)~) = \begin{cases}  T \quad &\text{when~} (x_3x_2x_1x_0)_{2,4} \textbf{ mod } 5 = 0 \\  F &\text{otherwise}\end{cases}$$
      \item $P_3: \{0,1\}\times \{0,1\} \times \{0,1\} \times \{0,1\} \to \{T, F\}$ given by 
      $$P_3(~(x_3,x_2,x_1,x_0)~) = \begin{cases}  T \quad &\text{when~} (x_3x_2x_1x_0)_{2,4}  = [x_3x_2x_1x_0]_{s,4} \\  F &\text{otherwise}\end{cases}$$
   \end{enumerate}
   \end{enumerate}

   \item Recall the functions \textit{mutation}, \textit{insertion}, and \textit{deletion} defined in class.
   We define the predicates:

   $Mut$ with domain $S \times S$ is defined by, for $s_1 \in S$ and $s_2 \in S$,
   \[
      Mut(~(s_1,s_2)~) = \exists k\in \mathbb{Z^+} \exists b \in B (~ mutation(~(s_1, k, b)~) = s_2~)
   \]
   $Ins$ with domain $S \times S$ is defined by, for $s_1 \in S$ and $s_2 \in S$,
   \[
      Ins(~(s_1,s_2)~) = \exists k\in \mathbb{Z^+} \exists b \in B (~ insertion(~(s_1, k, b)~) = s_2~)
   \]
   $Del$ with domain $\{ s\in S \mid rnalen(s) > 1\}  \times S$ is defined by, for 
   $s_1 \in \{ s\in S \mid rnalen(s) > 1\} $ and $s_2 \in S$,
   \[
      Del(~(s_1,s_2)~) = \exists k\in \mathbb{Z^+} (~ deletion(~(s_1, k)~) = s_2~)
   \]

   For each quantified statement below, {\bf first} translate to an English sentence.

   {\bf Then}, negate the {\bf whole} statement and rewrite this
   negated statement so that negations appear only within predicates 
   (that is, so that no negation is outside a quantifier or an expression involving logical connectives).
 
    The translations are graded for fair effort completeness.
 
   The negations are graded for correctness. For negations: You do not need to justify 
   your work for this part of the question.  However, if you include correct
   intermediate steps, we might be able to award partial credit for an incorrect answer.
 
  \rule{0.5\textwidth}{.4pt}
 
 {\it Sample response that can be used as reference for the detail expected 
 in your answer:} 
 Consider the statement
 \[
  \hspace{-1in}\forall n \in \mathbb{Z}^+~ \exists s \in S~\left( ~L(~(s,n)~) \land F_\A (s) \land BC(~(s,\A,n)~)~\right)
  \]
  
 {\bf Solution}: English translation is 
 \begin{quote}
 For each positive integer there is some RNA strand of that length that starts 
 with \A~ and all of its bases are \A.
 \end{quote} 
 
 We obtain the negation using multiple applications of De Morgan's rule and logical equivalences. 
 \begin{align*}
 \lnot &\forall n \in \mathbb{Z}^+~ \exists s \in S~\left( ~L(~(s,n)~) \land F_\A (s) \land BC(~(s,\A,n)~)~\right) \\
 \equiv&\exists n \in \mathbb{Z}^+~ \lnot \exists s \in S~\left( ~L(~(s,n)~) \land F_\A (s) \land BC(~(s,\A,n)~)~\right) \\
 \equiv&\exists n \in \mathbb{Z}^+~ \forall s \in S~\lnot \left( ~L(~(s,n)~) \land F_\A (s) \land BC(~(s,\A,n)~)~\right) \\
 \equiv&\exists n \in \mathbb{Z}^+~ \forall s \in S~\left( ~\lnot L(~(s,n)~) \lor \lnot F_\A (s) \lor \lnot BC(~(s,\A,n)~)~\right)
 \end{align*}
 
 
 \rule{0.5\textwidth}{.4pt}
 
 
 \begin{enumerate}
 \item  First statement:
 \[
 \forall s \in S ~\left( ~Mut(~(s,s)~) \leftrightarrow Ins(~(s,s)~) ~\right)
 \]
 
 \item Second statement
 \[
 \forall s_1 \in S ~ \forall s_2 \in S ~\forall s_3 \in S ~\left( ~\left(~Mut(~(s_1,s_2)~) \land Mut(~(s_2, s_3)~) ~\right) \to Mut(~(s_1,s_3)~)~\right)
 \]
 
 \item Third statement: we use $S'$ to abbreviate $\{ s\in S \mid rnalen(s) > 1\}$
 \[
 \forall s_2 \in S' ~\exists s_1 \in S'~\left(~ Del(~(s_1,s_2)~)~\right) 
 \land \lnot \forall s_1 \in S' ~\exists s_2 \in S'~\left(~ Del(~(s_1,s_2)~)~\right) 
 \]
 \end{enumerate}

 {\it Bonus; not for credit (do not hand in)}:  For each statement above, is the statement or its negation true? How do you know?
 
 
\end{enumerate}
\newpage

\title{HW4 Proofs and Sets}
\date{Due: Tuesday, November 2, 2021 at 11:00PM on Gradescope}


\maketitle
\thispagestyle{fancy}

{\bf In this assignment,}

You will analyze statements and determine if they are true or false using valid proof strategies.
You will also determine if candidate arguments are valid.

Instructions and academic integrity reminders for all homework assignments in 
CSE20 this quarter are on the class website and on the hw1-definitions-and-notations
assignment.

You will submit this assignment via Gradescope
(\href{https://www.gradescope.com}{https://www.gradescope.com}) 
in the assignment called ``hw4-proofs-and-sets''.

{\bf Resources}: To review the topics you are working with 
for this assignment, see the class material from Week 5.
We will post frequently asked questions and our answers to them in a 
pinned Piazza post.


\newpage
In your proofs and disproofs of statements below, justify each  step
by reference to  a component of the  following proof  strategies
we  have discussed so far, and/or to relevant definitions and calculations.
\begin{itemize}
    \item A counterexample can be used to prove that  $\forall x P(x)$ is {\bf false}.
    \item  A witness can be used to prove that  $\exists x P(x)$ is {\bf true}.
    \item {\bf Proof of universal by exhaustion}: To prove that $\forall x \, P(x)$
is true when $P$ has a finite domain, evaluate the predicate at {\bf each} domain element to confirm that it is always T.
    \item  {\bf Proof by universal generalization}: To prove that $\forall x \, P(x)$
is true, we can take an arbitrary element $e$ from the domain and show that $P(e)$ is true, without making any assumptions 
about $e$ other than that it comes from the domain.
    \item To  prove  that $\exists x P(x)$ is {\bf false}, write the universal statement that is 
    logically equivalent to its negation and then prove it true using universal generalization.
    \item {\bf Strategies for conjunction}: To prove that $p \land q$ is true, have two subgoals: 
    subgoal (1) prove $p$ 
is  true; and, subgoal (2) prove $q$ is true. To prove that $p \land q$ is false, it's enough to prove that $p$ is false.
 To prove that $p \land q$ is false, it's enough to prove that $q$ is false.
    \item {\bf Proof of Conditional by Direct Proof}: To prove that the implication $p \to q$ is true, 
    we can assume $p$ is true and use that assumption to show $q$ is true.
    \item {\bf Proof of Conditional by Contrapositive Proof}: To prove that the implication $p \to q$ is true, 
    we can assume $\neg q$ is true and use that assumption to show $\neg p$ is true.
    \item {\bf Proof of disjuction using equivalent conditional}: To prove that the 
    disjunction $p \lor q$ is true, we can rewrite it equivalently as $\lnot p \to q$ and
    then use direct proof or contrapositive proof.
    \item {\bf Proof by Cases}: To prove $q$ when we know $p_1 \lor p_2$, show that $p_1 \to q$ and $p_2 \to q$.
\end{itemize}

{\bf Assigned questions}

\begin{enumerate}
   \item Consider the predicate $Pr(x)$ over the set of integers, which evaluates to $T$ exactly when 
   $x$ is prime. Consider the following statements.
   
    \begin{multicols}{2}
    \begin{enumerate}[label=(\roman*)]
        \item $\exists x \in \mathbb{Z}~ \forall y \in \mathbb{Z}~(~x \leq y \to Pr(y)~)$
        \item $\exists x \in \mathbb{Z}~ \forall y \in \mathbb{Z}~(~y \leq x \to Pr(y)~)$
        \item $\forall x \in \mathbb{Z}~ \exists y \in \mathbb{Z}~(~x \leq y \to Pr(y)~)$
        \item $\forall x \in \mathbb{Z}~ \exists y \in \mathbb{Z}~(~y \leq x \to Pr(y)~)$
        \item $\exists x \in \mathbb{Z}~ \forall y \in \mathbb{Z}~(~Pr(y) \to y \leq x~)$
        \item $\exists x \in \mathbb{Z}~ \forall y \in \mathbb{Z}~(~Pr(y) \to x \leq y~)$
        \item $\forall x \in \mathbb{Z}~ \exists y \in \mathbb{Z}~(~Pr(y) \to y \leq x~)$
        \item $\forall x \in \mathbb{Z}~ \exists y \in \mathbb{Z}~(~Pr(y) \to x \leq y~)$
    \end{enumerate}
    \end{multicols}
   
   \begin{enumerate}
   
   \item ({\it Graded for correctness of choice and fair effort completeness in justification
   \footnote{Graded for correctness means your solution will be
   evaluated not only on the correctness of your answers, but on your ability to 
   present your ideas clearly and logically. You should explain how you arrived at your conclusions, using 
   mathematically sound reasoning. Whether you use formal proof techniques or write a more informal argument for why 
   something is true, your answers should always be well-supported. Your goal should be to convince the reader that 
   your results and methods are sound. Graded for fair effort completeness means 
   you will get full credit so long as your submission demonstrates honest 
   effort to answer the question. You will not be penalized for incorrect answers.}}) 
   Which of the statements (i) - (viii) is being {\bf proved} by the following proof:
   \begin{quote}
     Choose $x = 1$, an integer, and we will work to show
     it is a {\bf witness} for the existential claim. By universal generalization, {\bf choose} $e$ to be an {\bf arbitrary} integer. 
     Towards a {\bf direct proof}, {\bf assume} that $Pr(e)$ holds. We {\bf WTS} that $1 \leq e$.
     By definition of the  predicate $Pr$, since $Pr(e)$ is true, $e > 1$. By definition of $\leq$, 
     this means that $1 \leq e$, as required and the claim has been proved. $\square$
   \end{quote}
   
   
   {\it Hint: it may be useful to 
   identify the key words in the proof that indicate proof strategies.}
   
   \item ({\it Graded for correctness of choice and fair effort completeness in justification}) 
   Which of the statements (i) - (viii) is being {\bf disproved} by the following proof:
   \begin{quote}
     To disprove the statement, we will prove the universal
     statement that is logically equivalent to its negation. 
     By universal generalization, {\bf choose} $e$ to be an {\bf arbitrary} integer. 
     We need to find a {\bf witness} integer $y$ such that $y \leq e$ and $\lnot Pr(y)$.
     Notice that $e > 1 \lor e \leq 1$ is true, and we proceed in a {\bf proof by cases}.
     {\bf Case 1}: Assume $e > 1$ and {\bf WTS} there is a witness integer $y$ such that
     $y \leq e$ and $\lnot Pr(y)$. Choose $y = 0$, an integer. Then, since by {\bf case assumption}
     $1 < e$, we have $y = 0 \leq 1 \leq e$.
     Moreover, since $y = 0$, $y > 1$ is false and so (by the definition of $Pr$), the predicate $Pr$
     evaluated at $y$ is false, as required to prove the {\bf conjunction} $y \leq e$ and $\lnot Pr(y)$. 
     {\bf Case 2}: 
     Assume $e \leq 1$ and {\bf WTS} there is a witness integer $y$ such that
     $y \leq e$ and $\lnot Pr(y)$. Choose $y = e-1$, an integer (because subtracting
     $1$ from the integer $e$ still gives an integer). By definition of subtraction, $y = e-1 \leq e$.
     Moreover, since by the {\bf case assumption} $y = e-1 \leq 1-1= 0$, $y > 1$ is false. Thus, 
    (by the definition of $Pr$), the predicate $Pr$
     evaluated at $y$ is false. We have proved the {\bf conjunction} $y \leq e$ and $\lnot Pr(y)$ as required.
     Since each case is complete, the proof by cases is complete and the original
     statement has been disproved.  $\square$
   \end{quote}
   
   {\it Hint: it may be useful to 
   identify the key words in the proof that indicate proof strategies.}

   \item ({\it Graded for correctness of evaluation of statement (is it true or false?)
   and fair effort completeness of the translation and of the proof}) 
    Translate the statement to English and then prove or disprove it
   $$\forall x \in \mathbb{Z}~ \forall y \in \mathbb{Z}~(~x \neq y \to (Pr(x) \lor Pr(y))~)$$

   \item ({\it Graded for correctness of evaluation of statement (is it true or false?) 
   and fair effort completeness of the translation and proof}) 
   Translate the statement to English and then prove or disprove it
   $$\left( ~\forall x \in \mathbb{Z} ~Pr(x)~\right) \oplus \left(~\exists x \in \mathbb{Z} ~Pr(x) ~\right)$$

   \item ({\it Graded for correctness of evaluation of statement (is it true or false?) 
   and fair effort completeness of the translation and of the proof}) 
    Translate the statement to English and then prove or disprove it
   $$\forall x \in \mathbb{Z}~ \forall y \in \mathbb{Z}~(~(~Pr(x) \land Pr(y)~) \leftrightarrow Pr(x+y)~)$$
   
   \item ({\it Graded for correctness of evaluation of statement (is it true or false?)
   and fair effort completeness of the translation and of the proof}) 
    Translate the statement to English and then prove or disprove it
   $$\forall x \in \mathbb{Z}~ (~Pr(x) \to \exists y \in \mathbb{Z}~(~x < y \land Pr(y)~)$$

   \end{enumerate}

   
   \item Let $W = \mathcal{P}(\{1,2,3,4,5\})$. 
   
   \rule{0.5\textwidth}{.4pt}
   
   {\it Sample response that can be used as reference for the detail expected 
   in your answers for this question:} 
   
   To give a witness for the existential claim
   $$ \exists B \in W~( B \in ~\{ X \in W ~|~ 1 \in X \} \cap \{ X \in W ~|~  2 \in X \}~~)$$
   consider $B = \{ 1,2\}$. To prove that this is a valid witness, we need
   to show that it is in the domain of quantification $W$ and that 
   it makes the predicate being quantified evaluate to true. By definition 
   of set-builder notation and intersection, it's enough to prove
   that $\{1,2\} \in W$ and that $1 \in \{1,2\}$ and that $2 \in \{1,2\}$.
   \begin{itemize}
   \item By definition of power set, elements of $W$ are subsets of $\{1,2,3,4,5\}$. Since
   each element in $\{1,2\}$ is an element of $\{1,2,3,4,5\}$, $\{1,2\}$ is a subset of $\{1,2,3,4,5\}$ 
   and hence is an element of $W$. 
   \item Also, by definition of the roster method, $1 \in \{1,2\}$. 
   \item Similarly, by definition of roster method, $2 \in \{1,2\}$.
   \end{itemize}
   Thus $B = \{1,2\}$ is an element of the domain which is in the intersection of the 
   two sets mentioned in the predicate being quantified and is a witness to the existential claim. QED
   
   \rule{0.5\textwidth}{.4pt}
   
   
   \begin{enumerate}
   \item ({\it Graded for correctness}) Give a witness to the existential claim
   $$ \exists X \in W ~(~X \cup X = \emptyset~)$$
   Justify your example by explanations that include references to the relevant definitions.
   
   \item ({\it Graded for correctness}) Give a counterexample to the universal claim
   $$ \forall X \in W ~( \{ a \in X \mid a \textrm{ is even} \} \subsetneq X~)$$
   Justify your example by explanations 
   that include references to the relevant definitions.
   
   \item  ({\it Graded for correctness}) Give a witness to the existential claim
   $$ \exists (X,Y) \in W \times W ~(~X \cup Y = Y~)$$
   Justify your example by explanations that include references to the relevant definitions.
   \end{enumerate}
   

   \item Recall our representation of movie preferences in a three-movie database 
   using $1$ in a component to indicate liking the movie represented by that component, 
   $-1$ to indicate not liking the movie, and $0$ to indicate neutral opinion or
   haven't seen the movie. We call $Rt$ the set of all ratings $3$-tuples. 
   We defined the function 
   $d_0: Rt\times Rt \to \mathbb{R}$ which takes an ordered pair of ratings $3$-tuples and returns a measure
   of the distance between them 
   given by
   \[
   d_0 (~(~ (x_1, x_2, x_3), (y_1, y_2, y_3) ~) ~) = \sqrt{ (x_1 - y_1)^2 + (x_2 - y_2)^2 + (x_3 -y_3)^2}
   \]
   Another measure of the distance between a pair of ratings $3$-tuples is given by 
   the following function $d_1: Rt\times Rt \to \mathbb{R}$ given by 
   \[
   d_1 (~(~ (x_1, x_2, x_3), (y_1, y_2, y_3) ~) ~) = \sum_{i=1}^3 |x_i - y_i|
   \]
   \begin{enumerate}
    \item    For each of the statements below, first translate them symbolically (using
        quantifiers, logical operators, and arithmetic operations), then determine whether each 
        is true or false by applying the proof strategies to prove each statement or its negation.
        ({\it Graded for correctness of evaluation of statement (is it true or false?) and 
        fair effort completeness of the translation and of the proof}) 
        \begin{enumerate}
            \item For all ordered pairs of ratings $3$-tuples, the value of the function $d_0$ 
            is greater than the value of the function $d_1$.
            \item The maximum value of the function $d_1$ is greater than the maximum value of the function $d_0$.
        \end{enumerate}

    \item ({\it Graded for correctness}) Write a statement about 3-tuples of movie ratings that uses the function 
    $d_1$ and has at least one universal and one existential quantifier. Your response will be 
    graded correct if all the syntax in your statement is correct.

    \item ({\it Graded for fair effort completeness}) Translate the property you wrote symbolically in the 
    last step to English. Indicate if it is true, false, or if you don't know 
    (sometimes we can write interesting properties, and we're not sure if they are true or not!). 
    Give informal justification for whether  you think it is true/ false, or explain why 
    the proof strategies we have so far do not appear to  be sufficient to determine whether the statement holds.
    \end{enumerate}

\end{enumerate}
\newpage

\title{HW5 Proofs and Induction}
\date{Due: Tuesday, November 16, 2021 at 11:00PM on Gradescope}


\maketitle
\thispagestyle{fancy}

{\bf In this assignment,}

You will work with recursively defined sets and functions and prove 
properties about them, practicing induction and other proof strategies.

Instructions and academic integrity reminders for all homework assignments in 
CSE20 this quarter are on the class website and on the hw1-definitions-and-notations
assignment.

You will submit this assignment via Gradescope
(\href{https://www.gradescope.com}{https://www.gradescope.com}) 
in the assignment called ``hw5-proofs-and-induction''.

{\bf Resources}: To review the topics you are working with 
for this assignment, see the class material from Weeks 5 through 7.
We will post frequently asked questions and our answers to them in a 
pinned Piazza post.


In your proofs and disproofs of statements below, justify each  step
by reference to  a component of the  following proof  strategies
we  have discussed so far, and/or to relevant definitions and calculations.
\begin{itemize}
    \item A counterexample can be used to prove that  $\forall x P(x)$ is {\bf false}.
    \item  A witness can be used to prove that  $\exists x P(x)$ is {\bf true}.
    \item {\bf Proof of universal by exhaustion}: To prove that $\forall x \, P(x)$
is true when $P$ has a finite domain, evaluate the predicate at {\bf each} domain element to confirm that it is always T.
    \item  {\bf Proof by universal generalization}: To prove that $\forall x \, P(x)$
is true, we can take an arbitrary element $e$ from the domain and show that $P(e)$ is true, without making any assumptions 
about $e$ other than that it comes from the domain.
    \item To  prove  that $\exists x P(x)$ is {\bf false}, write the universal statement that is 
    logically equivalent to its negation and then prove it true using universal generalization.
    \item {\bf Strategies for conjunction}: To prove that $p \land q$ is true, have two subgoals: 
    subgoal (1) prove $p$ 
is  true; and, subgoal (2) prove $q$ is true. To prove that $p \land q$ is false, it's enough to prove that $p$ is false.
 To prove that $p \land q$ is false, it's enough to prove that $q$ is false.
    \item {\bf Proof of Conditional by Direct Proof}: To prove that the implication $p \to q$ is true, 
    we can assume $p$ is true and use that assumption to show $q$ is true.
    \item {\bf Proof of Conditional by Contrapositive Proof}: To prove that the implication $p \to q$ is true, 
    we can assume $\neg q$ is true and use that assumption to show $\neg p$ is true.
    \item {\bf Proof of disjuction using equivalent conditional}: To prove that the 
    disjunction $p \lor q$ is true, we can rewrite it equivalently as $\lnot p \to q$ and
    then use direct proof or contrapositive proof.
    \item {\bf Proof by Cases}: To prove $q$ when we know $p_1 \lor p_2$, show that $p_1 \to q$ and $p_2 \to q$.
    \item
    {\bf Proof by Structural Induction}: To prove that $\forall x \in X \, P(x)$ where $X$ is a recursively defined set, prove two cases:
        
        \begin{tabularx}{\textwidth}{l X}
        Basis Step: & Show the statement holds for elements specified in the basis step of the definition. \\
        Recursive Step: & Show that if the statement is true for each of the elements used to construct
    new elements in the recursive step of the definition, the result holds for these new elements.
    \end{tabularx}
    
    \item {\bf Proof by Mathematical Induction}: To prove a universal quantification over the set of  all integers greater than  or  equal to some base integer $b$:
    
    \begin{tabularx}{\textwidth}{l X}
        Basis Step: & Show the statement holds for $b$. \\
        Recursive Step: & Consider an arbitrary integer $n$ greater than or  equal to  $b$, assume
        (as the {\bf induction hypothesis})  that the property holds  for $n$, and use  this and
        other facts to  prove that  the property holds for $n+1$.
    \end{tabularx}
    
    \item {\bf Proof by Strong Induction} To prove that a universal quantification over the set of all integers greater than or equal to some  base integer $b$ holds,  pick a  fixed nonnegative integer  $j$ and then: \hfill 
    
    \begin{tabularx}{\textwidth}{l X}
        Basis Step: & Show the statement holds for $b$, $b+1$, \ldots, $b+j$. \\
        Recursive Step: & Consider an arbitrary integer $n$ greater than or  equal to  $b+j$, assume
        (as the {\bf strong  induction hypothesis})  that the property holds  for {\bf each of} $b$, $b+1$, \ldots, $n$, 	
        and use  this and
        other facts to  prove that  the property holds for $n+1$.
    \end{tabularx}

    \item {\bf Proof by Contradiction} 

    To prove that a statement $p$ is true, pick another statement $r$ and once we show
    that $\neg p  \to (r \wedge  \neg r)$ then  we can conclude that  $p$ is  true.
    
    {\it Informally} The statement we care about can't possibly be false, so it must be true.
\end{itemize}

\newpage
{\bf Assigned questions}

\begin{enumerate}
   \item Recall the definitions from class about factoring and divisibility:
   when $a$ and $b$ are integers and $a$ is nonzero, 
   {\bf $a$ divides $b$} means there is an integer $c$ such that $b = ac$ . 
   In this case, we say $a$ is a {\bf factor} of $b$, $a$ is a {\bf divisor} of $b$, 
   $b$ is a {\bf multiple} of $a$, 
   $a | b$.  We define the function 
   $PosFactors: \mathbb{Z}^+ \to \mathcal{P}(\mathbb{Z}^+)$ by 
   \[
        PosFactors (n) = \{ x \in \mathbb{Z}^+ \mid x\text{ is a factor of } n\}
   \]

   \rule{0.5\textwidth}{.4pt}

   {\it Sample calculation that can be used as reference for the detail expected 
   in your answer when working with this function:} 
   
   The function application $PosFactors(4)$ evaluates to 
   \[
       PosFactors (4) = \{ 1,2,4\}
   \]
   because the only possible positive factors of $4$ are $1,2,3,4$ (the positive integers less than 
   or equal to $4$) and when we divide we get:
   \begin{align*}
        4 &= 4 \cdot 1 + 0 \qquad \text{so $4$ is a factor of $4$}\\
        4 &= 3 \cdot 1 + 1 \qquad \text{so $3$ is not a factor of $4$}\\
        4 &= 2 \cdot 2 + 0 \qquad \text{so $2$ is a factor of $4$}\\
        4 &= 1 \cdot 4 + 0 \qquad \text{so $1$ is a factor of $4$}
   \end{align*}

   \rule{0.5\textwidth}{.4pt}

   \begin{enumerate}
     \item ({\it Graded for correctness}\footnote{Graded for correctness means your solution will be
     evaluated not only on the correctness of your answers, but on your ability to 
     present your ideas clearly and logically. You should explain how you arrived at your conclusions, using 
     mathematically sound reasoning. Whether you use formal proof techniques or write a more informal argument for why 
     something is true, your answers should always be well-supported. Your goal should be to convince the reader that 
     your results and methods are sound.}) Give a witness that proves the statement 
     \[
         \exists x \in \mathbb{Z}^+ ~\forall y \in \mathbb{Z}^+ ~\left(~x \in PosFactors(y)~\right)
     \]
     Justify your choice of witness by explanations that include references to the relevant definitions.
     \item ({\it Graded for correctness}) Give a counterexample that disproves the statement 
     \[
         \forall n \in \mathbb{Z}^+ ~\left(~ PosFactors(n) \subseteq PosFactors(n+1)~\right)
     \]
     Justify your choice of counterexample by explanations that include references to the relevant definitions.
    
    \item ({\it Graded for fair effort completeness}) Consider the following attempted proof.
    \begin{quote}
    {\bf Attempted proof}: For arbitrary integers $a, b, c$, assume towards a direct proof that 
    $(a+b) | c$.  We need to show that $a|c$
    and $b | c$. Let $n$ be the integer $c \text{\bf ~div~} (a+b)$. 
    Since $(a+b) | c$, by definition of divides, $n | c$
    and $n$ is an integer.  Since $c = 1 \cdot n \cdot (a+b)$, $(n \cdot (a+b) ) | c$. 
    Rewriting by distributing 
    multiplication over addition, 
    we have $na | c$ and $nb | c$. Since $a | na$ and $na | c$, we have $a | c$.  
    Similarly, since $b | nb$ and 
    $nb | c$, we have $b | c$.  Thus, we have proved both conjuncts and the proof is complete.
    \end{quote}
    Select the statement below that the attempted proof is trying to prove.  
    \begin{enumerate}[label=(\roman*)]
    \item $\forall a \in \mathbb{Z}^+ ~\forall b \in \mathbb{Z}^+ ~\forall c \in \mathbb{Z} ~( ~( ~a|c ~\lor~b |c~) ~\to~ (a+b) |c~)$
    \item $\forall a \in \mathbb{Z}^+ ~\forall b \in \mathbb{Z}^+ ~\forall c \in \mathbb{Z} ~( ~( ~a|b ~\land~a |c~) ~\to~ a |(b+c)~)$
    \item $\forall a \in \mathbb{Z}^+ ~\forall b \in \mathbb{Z}^+ ~\forall c \in \mathbb{Z} ~(~ (a+b) | c ~\to ~( ~a|c ~\land~b |c~)~)$
    \end{enumerate}
    
    Identify the first major error in the attempted proof and explain why it is incorrect.
    
    Next, disprove the statement the attempted proof was attempting to prove.
    
    {\it Extra practice; not for credit}: prove or disprove the other two statements.
    \end{enumerate}

    \item In this question, we'll consider the function which calculates the sum of the first $n$ positive integers.
   \begin{enumerate}
        \item ({\it Graded for fair effort completeness}\footnote{Graded for fair effort completeness means 
        you will get full credit so long as your submission demonstrates honest 
        effort to answer the question. You will not be penalized for incorrect answers.}) 

        Give a recursive definition of this function, including domain, codomain and both the basis step
        and recursive step of the rule. That is, fill in the blanks 
        \[
            sumOfFirst: \underline{~~domain~~} \to \underline{~~codomain}
        \]
        given by 
        \begin{align*}
            &\textbf{Basis step}: \underline{\text{fill in basis step}} \\
            &\textbf{Recursive step}: \underline{\text{fill in recursive step}}
        \end{align*}

        {\it Notation}: Using summation, this function can be written $sumOfFirst(n) = \sum_{i=1}^n i$.\\


        \item ({\it Graded for fair effort completeness}) It turns out that the value of this function
        can also be calculated explicitly (without recursion)\footnote{When the value of a function 
        that is recursively defined can also be calculated without recursion, we call the formula 
        that we can use to calculate the value without recursion the ``closed form formula'' for the 
        function.}. You will 
        prove this by completing the proof of the identity
        \[
            \forall n \in \mathbb{Z}^+ ~\left(~sumOfFirst(n) = \frac{n(n+1)}{2} \right)
        \]
        
        Fill in the missing parts of the proof of this statement:\\

        {\bf Proof}: We proceed by mathematical induction on the set of positive integers.

        {\bf Basis Step}: Choose $n = 1$ as the basis step. 
        Using the Basis Step in the recursive definiton of $sumOfFirst$, 
        $sumOfFirst(1) = 1$. Plugging $n$ into the RHS of the desired formula,
        $\frac{1 (1+1)}{2} = \frac{2}{2} = 1$. Since LHS=RHS, the Basis step is complete.
   
        {\bf Recursive Step}: Consider an arbitrary $k \geq 1$.  
        We assume (as the induction hypothesis) that \underline{fill in the blank here}. 
   
        We want to show that $sumOfFirst(k+1) = \frac{(k + 1) \cdot ((k + 1) + 1)}{2}$.
   
        \underline{Fill in the rest of the proof here.} \\

        \item ({\it Graded for fair effort completeness})
        When calculating the runtime of an algorithm, nested for loops sometimes lead to program 
        runtimes that involve the sum of the first $n$ positive integers. To estimate
        the rate of growth of this runtime, it is useful to find an upper bound for this function 
        in terms of a simpler function.  Use the explicit formula from the earlier parts of this question
        and mathematical induction to prove
        \[
            \forall n \in \mathbb{Z}^+~ \left(~sumOfFirst(n) \leq n^2 \right)
        \]

   \end{enumerate}

   \item Recall the definition of linked lists that we discussed in class.


   Define the function $count$ which returns the number of occurrences of a datum
   in the list. Formally, $count: L \times \mathbb{N} \to \mathbb{N}$, where
   \begin{align*}
    \textbf{Basis Step:} \qquad &\textrm{If } m \in \mathbb{N},~~ count(~( [], m)~ ) = 0 \\ 
    \textbf{Recursive Step:} \qquad &\textrm{If } l \in L\textrm{ and }n \in \mathbb{N}
    \textrm{ and }m \in \mathbb{N} \textrm{, then  } \\
    &count(~(~(n, l),m~)~ ) = 
        \begin{cases}
            1 + count(~(l,m)~) &\text{if $n=m$} \\
            count(~(l,m)~) &\text{otherwise} \\
        \end{cases}
    \end{align*}

    A mystery function is defined by 
    $mystery : L \times \mathbb{N} \to L$, where

    \begin{align*}
    \textbf{Basis Step:} \qquad &\textrm{If } m \in \mathbb{N},~~ mystery(~( [], m)~ ) = [] \\ 
    \textbf{Recursive Step:} \qquad &\textrm{If } l \in L\textrm{ and }n \in \mathbb{N}
    \textrm{ and }m \in \mathbb{N} \textrm{, then  } \\
    &mystery(~(~(n, l),m~)~ ) = 
        \begin{cases}
            l &\text{if $n=m$} \\
            mystery(~(l,m)~) &\text{otherwise} \\
        \end{cases}
    \end{align*}

   \begin{enumerate}
        \item ({\it Graded for correctness}) Prove that
            \[
                \forall m \in \mathbb{N}~ \exists l \in L ~\left( count(~(l,20)~) = m~ \right)
            \]

        \item ({\it Graded for correctness}) Give an example input $x$ to the function such that 
            \[
               mystery( ~(~ x ~,~ 2~) ~) = []
            \]
           For full credit, include all intermediate steps of the function application
           that justifies your choice of $x$, with brief justifications for each.

        \item ({\it Graded for correctness}) Evaluate the function application
            \[
                mystery( ~(~ (2, (0, (2, []) ) ) ~,~ 2~) ~)
            \]
         For full credit, include all intermediate steps of the function application,
         with brief justifications for each.
   
        \item ({\it Graded for fair effort completeness for English statements and correctness in use 
        of syntax for symbolic statements}) Describe the rule of the function 
        $mystery$ in English. Then, write a true statement that describes an invariant using
        both the functions $mystery$ and $count$. Express this invariant both symbolically 
        and in English.
    \end{enumerate}

\end{enumerate}

    
\newpage

\title{HW6 Proofs, Numbers, and Cardinality}
\date{Due: Tuesday, November 23, 2021 at 11:00PM on Gradescope}


\maketitle
\thispagestyle{fancy}

{\bf In this assignment,}

You will practice determining and justifying whether 
statements are true in multiple contexts.

Instructions and academic integrity reminders for all homework assignments in 
CSE20 this quarter are on the class website and on the hw1-definitions-and-notations
assignment.

You will submit this assignment via Gradescope
(\href{https://www.gradescope.com}{https://www.gradescope.com}) 
in the assignment called ``hw6-proofs-numbers-cardinality''.

{\bf Resources}: To review the topics you are working with 
for this assignment, see the class material from Weeks 6 through 8.
We will post frequently asked questions and our answers to them in a 
pinned Piazza post.


In your proofs and disproofs of statements below, justify each  step
by reference to  a component of the  following proof  strategies
we  have discussed so far, and/or to relevant definitions and calculations.
\begin{itemize}
    \item A counterexample can be used to prove that  $\forall x P(x)$ is {\bf false}.
    \item  A witness can be used to prove that  $\exists x P(x)$ is {\bf true}.
    \item {\bf Proof of universal by exhaustion}: To prove that $\forall x \, P(x)$
is true when $P$ has a finite domain, evaluate the predicate at {\bf each} domain element to confirm that it is always T.
    \item  {\bf Proof by universal generalization}: To prove that $\forall x \, P(x)$
is true, we can take an arbitrary element $e$ from the domain and show that $P(e)$ is true, without making any assumptions 
about $e$ other than that it comes from the domain.
    \item To  prove  that $\exists x P(x)$ is {\bf false}, write the universal statement that is 
    logically equivalent to its negation and then prove it true using universal generalization.
    \item {\bf Strategies for conjunction}: To prove that $p \land q$ is true, have two subgoals: 
    subgoal (1) prove $p$ 
is  true; and, subgoal (2) prove $q$ is true. To prove that $p \land q$ is false, it's enough to prove that $p$ is false.
 To prove that $p \land q$ is false, it's enough to prove that $q$ is false.
    \item {\bf Proof of Conditional by Direct Proof}: To prove that the implication $p \to q$ is true, 
    we can assume $p$ is true and use that assumption to show $q$ is true.
    \item {\bf Proof of Conditional by Contrapositive Proof}: To prove that the implication $p \to q$ is true, 
    we can assume $\neg q$ is true and use that assumption to show $\neg p$ is true.
    \item {\bf Proof of disjuction using equivalent conditional}: To prove that the 
    disjunction $p \lor q$ is true, we can rewrite it equivalently as $\lnot p \to q$ and
    then use direct proof or contrapositive proof.
    \item {\bf Proof by Cases}: To prove $q$ when we know $p_1 \lor p_2$, show that $p_1 \to q$ and $p_2 \to q$.
    \item
    {\bf Proof by Structural Induction}: To prove that $\forall x \in X \, P(x)$ where $X$ is a recursively defined set, prove two cases:
        
        \begin{tabularx}{\textwidth}{l X}
        Basis Step: & Show the statement holds for elements specified in the basis step of the definition. \\
        Recursive Step: & Show that if the statement is true for each of the elements used to construct
    new elements in the recursive step of the definition, the result holds for these new elements.
    \end{tabularx}
    
    \item {\bf Proof by Mathematical Induction}: To prove a universal quantification over the set of  all integers greater than  or  equal to some base integer $b$:
    
    \begin{tabularx}{\textwidth}{l X}
        Basis Step: & Show the statement holds for $b$. \\
        Recursive Step: & Consider an arbitrary integer $n$ greater than or  equal to  $b$, assume
        (as the {\bf induction hypothesis})  that the property holds  for $n$, and use  this and
        other facts to  prove that  the property holds for $n+1$.
    \end{tabularx}
    
    \item {\bf Proof by Strong Induction} To prove that a universal quantification over the set of all integers greater than or equal to some  base integer $b$ holds,  pick a  fixed nonnegative integer  $j$ and then: \hfill 
    
    \begin{tabularx}{\textwidth}{l X}
        Basis Step: & Show the statement holds for $b$, $b+1$, \ldots, $b+j$. \\
        Recursive Step: & Consider an arbitrary integer $n$ greater than or  equal to  $b+j$, assume
        (as the {\bf strong  induction hypothesis})  that the property holds  for {\bf each of} $b$, $b+1$, \ldots, $n$, 	
        and use  this and
        other facts to  prove that  the property holds for $n+1$.
    \end{tabularx}

    \item {\bf Proof by Contradiction} 

    To prove that a statement $p$ is true, pick another statement $r$ and once we show
    that $\neg p  \to (r \wedge  \neg r)$ then  we can conclude that  $p$ is  true.
    
    {\it Informally} The statement we care about can't possibly be false, so it must be true.
\end{itemize}

\newpage
{\bf Assigned questions}

\begin{enumerate}
    \item Recall the definition of the set of linked lists from class, and some associated functions.
    \[
        \begin{array}{ll}
        \textrm{Basis Step: } & [] \in L \\
        \textrm{Recursive Step: } & \textrm{If } l \in L\textrm{ and }n \in \mathbb{N} \textrm{, then } (n, l) \in L
        \end{array}
    \]
    The length function $length: L \to \mathbb{N}$ is defined by
    \[
        \begin{array}{llll}
        \textrm{Basis Step:} &  & length(~[]~) &= 0 \\
        \textrm{Recursive Step:} & \textrm{If } l \in L\textrm{ and }n \in \mathbb{N}\textrm{, then  } & length(~(n, l)~)  &= 1+ length(l)
        \end{array}
    \]
    The function $prepend : L \times \mathbb{N} \to L$ is defined by
    \[
        prepend(~(l, n)~) = (n, l)
    \]
    The function $append : L \times \mathbb{N} \to L$ is defined by
    \[
    \hspace{-40pt}\begin{array}{llll}
    \textrm{Basis Step:} & \textrm{If } m \in \mathbb{N}\textrm{ then } & append(~([], m)~) & = (m, []) \\
    \textrm{Recursive Step:} & \textrm{If } l \in L\textrm{ and }n \in \mathbb{N}\textrm{ and }m \in \mathbb{N}\textrm{, then  } & append(~(~(n, l), m~)~) &= (n, append(~(l, m)~)~)
    \end{array}
    \]
    \begin{enumerate}
        \item ({\it Graded for fair effort completeness}\footnote{Graded for fair effort completeness means 
        you will get full credit so long as your submission demonstrates honest 
        effort to answer the question. You will not be penalized for incorrect answers.}) Fill in the blanks in the following proof of the statement
        \[
            \forall l\in L~ \forall m \in \mathbb{N} ~(~length(prepend(~(l,m)~)) = length(append(~(l,m)~))~)
        \]
        {\bf Proof}: We proceed by structural induction on $L$.
        
        {\bf Basis step}: We need to show that 
        \[
            \forall m \in \mathbb{N}  ~(~length(prepend(~([],m)~)) = length(append(~([],m)~))~)
        \]
        Towards universal generalization, let $m$ be an arbitrary natural number. Calculating:
        \[
            LHS = \underline{\text{BLANK~1}}
        \]
        \[
            RHS = \underline{\text{BLANK~2}}
        \]
        Since $LHS = RHS$, the basis step is complete.

        {\bf Recursive step}: Consider an arbitrary: $l' \in L$, $n \in \mathbb{N}$, and we  assume
        as the {\bf induction hypothesis} that:
        \[
            \underline{\text{BLANK~3}}
        \]
        Our goal is to show that 
        \[
            \forall m \in \mathbb{N} ~(~length(prepend(~(~(n,l')~,m)~)) = length(append(~(~(n,l'),m)~))~)
        \]
        is also true. Let $m$ be an arbitrary natural number. Calculating: 
        \[
            LHS = \underline{\text{BLANK~4}}
        \]
        \[
            RHS = \underline{\text{BLANK~5}}
        \]
        Since $LHS = RHS$, the recursive step is complete.

        \item ({\it Graded for correctness}\footnote{Graded for correctness means your solution will be
        evaluated not only on the correctness of your answers, but on your ability to 
        present your ideas clearly and logically. You should explain how you arrived at your conclusions, using 
        mathematically sound reasoning. Whether you use formal proof techniques or write a more informal argument for why 
        something is true, your answers should always be well-supported. Your goal should be to convince the reader that 
        your results and methods are sound.}) Disprove the statement 
        \[
            \forall l\in L~ \forall m \in \mathbb{N} ~(~prepend(~(l,m)~) = append(~(l,m)~)~)
        \]

        \item ({\it Graded for correctness}) Determine whether the statement
        \[
            \exists l\in L~ \exists m \in \mathbb{N} ~(~prepend(~(l,m)~) = append(~(l,m)~)~)
        \]
        is true or false, and justify your conclusion using valid proof strategies.
    \end{enumerate}

    \item Recall the definition of the set of rational numbers,
    $$Q = \left\{ \frac{p}{q} \mid p \in \mathbb{Z}  \text{ and  } q  \in \mathbb{Z} \text{ and } q \neq  0 \right\}$$
    We define the set of {\bf irrational} numbers $\overline{\mathbb{Q}} = \mathbb{R} - \mathbb{Q}
    = \{ x \in \mathbb{R} \mid x \notin \mathbb{Q} \}$. 

    \begin{enumerate}
        \item ({\it Translation graded for fair effort completeness; Witness graded 
        for correctness}) Translate the statement to English and then give a witness 
        that could be used to prove the statement 
        \[
        \exists x \in \mathbb{Q}~ \forall y \in \overline{\mathbb{Q}} ~(x\cdot y \in \mathbb{Q})
        \]
        You do not need to justify your answer.  However, if you include clear explanations, 
        we may be able to give partial credit for an answer with some imprecision.
   
        \item ({\it Translation graded for fair effort completeness; Counterexample graded 
        for correctness}) Translate the statement to English and then give a counterexample 
        that could be used to disprove the statement
        \[
        \forall x \in \overline{\mathbb{Q}} ~\left(  x > 0 ~\to~ x \geq 1 \right)
        \]
        You do not need to justify your answer.  However, if you include clear explanations, 
        we may be able to give partial credit for an answer with some imprecision.
   
        \item ({\it Translation graded for fair effort completeness; Witness graded 
        for correctness}) Translate the statement to English and then give a witness that 
        could be used to prove the statement 
        \[
        \exists (x,y,z) \in \overline{\mathbb{Q}}\times\mathbb{Q}\times \mathbb{Q} ~(y \neq z \land x^ y= z)
        \]
        You do not need to justify your answer.  However, if you include clear explanations, 
        we may be able to give partial credit for an answer with some imprecision.
   
        \item ({\it Graded for fair effort completeness}\footnote{Fair effort completeness
        for this question means either attempting to correctly answer each part
        or to write a sentence or two on where you get stuck in your attempt to correctly answer
        the question.}) Fill in the blanks in the following argument.

        {\bf Claimed statement}: 
        $\exists x \in \overline{\mathbb{Q}}~ \exists y \in \overline{\mathbb{Q}} 
        ~\exists z\in \mathbb{Q} ~(x^y = z)$.
        
        \begin{quote}
        {\bf Proof}: We need to give a witness to prove this existential claim. 
        We proceed in a proof by cases, since the disjunction
        \textbf{(i)}$\underline{\phantom{\hspace{1.3in}}}$ is true.
        \begin{itemize}
        \item {\bf Case 1}: We need to show that 
        \[
            (\sqrt{2}^{\sqrt{2}} \in \mathbb{Q}) \to 
            \exists x \in \overline{\mathbb{Q}}~ \exists y \in \overline{\mathbb{Q}} 
            ~\exists z\in \mathbb{Q} ~(x^y = z)
        \]
        Assume towards a direct proof that \textbf{(ii)}$\underline{\phantom{\hspace{1.3in}}}$. We
        choose the witnesses $x = \sqrt{2}$, $y = \sqrt{2}$, $z = \sqrt{2}^{\sqrt{2}}$.  By the 
        theorem we proved in class, $\sqrt{2} \notin \mathbb{Q}$.
        Since $x = y = \sqrt{2}$, $x \in \overline{\mathbb{Q}}$ and  $y \in \overline{\mathbb{Q}}$.
        By the assumption of this direct proof, $z \in \mathbb{Q}$. Thus, the witnesses we picked
        are in the required domains.  Moreover, by definition, $z = x^y$, as required. Thus, the existential claim
        is proved and we have completed the direct proof required for this case.
   
        \item {\bf Case 2}: We need to show that 
        \[
            (\sqrt{2}^{\sqrt{2}} \in \overline{\mathbb{Q}}) \to 
            \exists x \in \overline{\mathbb{Q}}~ \exists y \in \overline{\mathbb{Q}} 
            ~\exists z\in \mathbb{Q} ~(x^y = z)
        \]
        Assume towards a direct proof that $(\sqrt{2}^{\sqrt{2}} \in \overline{\mathbb{Q}})$. We
        choose the witnesses 
        \[
        x = \textbf{(iii)}\underline{\phantom{\hspace{1.3in}}}, ~y = 
        \textbf{(iv)}\underline{\phantom{\hspace{1.3in}}}, ~z = \textbf{(v)}\underline{\phantom{\hspace{1.3in}}}
        \]  
        By the assumption of this direct proof, $x \in \overline{\mathbb{Q}}$. As we mentioned above, $\sqrt{2} \notin \mathbb{Q}$ so $y \in \overline{\mathbb{Q}}$. Picking $p = 2, q = 1$, we observe that $z = \frac{2}{1}$
        and since \textbf{(vi)}\underline{\phantom{\hspace{1.3in}}}, $z \in \mathbb{Q}$.  Thus, the 
        three witnesses we picked are in the required domains. 
        Calculating: 
        \[
        \left( \sqrt{2}^{\sqrt{2}}\right)^{\sqrt{2}} = \left( \sqrt{2} \right)^{\sqrt{2} \cdot \sqrt{2} }
        = \left(\sqrt{2} \right)^2 = \sqrt{2} \cdot \sqrt{2} = 2
        \]
        which proves that $\textbf{(vii)}\underline{\phantom{\hspace{1.3in}}}$, and hence $x,y,z$ are 
        the required witnesses. Thus, the existential claim
        is proved and we have completed the direct proof required for this case.
        \end{itemize}
        The proof by cases is now complete and the statement has been proved. QED
        \end{quote}

   \end{enumerate}
   
   \item Recall that  A {\bf hex color} is a nonnegative
   integer, $n$, that has a base $16$ fixed-width $6$ expansion
   $$n = (r_1r_2g_1g_2b_1b_2)_{16,6}$$ 
   where $(r_1r_2)_{16,2}$ is the red
   component, $(g_1g_2)_{16,2}$ is the green component, and $(b_1b_2)_{16,2}$ is the
   blue component.   For notational convenience, we define the set 
   $C = \{ x \in \mathbb{N} ~\mid~x  < 16^6 \}$.  This is the set of possible hex colors because these
   are all numbers that have hexadecimal fixed-width $6$ expansions. 
   \begin{enumerate}
    \item ({\it Graded for correctness}) Determine and briefly justify whether $C$ is finite, countably infinite, or uncountable.
    \item Consider the function $red: C \to C$ given by $red(~(r_1r_2g_1g_2b_1b_2)_{16,6}~) = (r_1r_20000)_{16,6}$.
        \begin{enumerate}
            \item ({\it Graded for correctness}) Determine whether $red$ is one-to-one, 
            and justify your conclusion using valid proof strategies.
            \item ({\it Graded for correctness}) Determine whether $red$ is onto, 
            and justify your conclusion using valid proof strategies.
        \end{enumerate}
   \end{enumerate}

    \item Consider the set of ratings in a 3-movie database 
    $R = \{ -1,0,1\} \times \{-1,0,1\} \times \{-1,0,1\}$ and the set of 
    bases of RNA strands $B = \{\A, \C, \U, \G\}$.
    \begin{enumerate}
        \item ({\it Graded for fair effort completeness}) Give a (well-defined) one-to-one function with domain $R$ and codomain $B$ or explain why there is no such function.
        \item ({\it Graded for fair effort completeness}) Give a (well-defined) one-to-one function with domain $B$ and codomain $R$ or explain why there is no such function.
        \item ({\it Graded for fair effort completeness}) Give a (well-defined) onto function with domain $R$ and codomain $B$ or explain why there is no such function.
        \item ({\it Graded for fair effort completeness}) Give a (well-defined) onto function with domain $B$ and codomain $R$ or explain why there is no such function.
    \end{enumerate}

    \rule{0.5\textwidth}{.4pt}

    {\it Sample calculation that can be used as reference for the detail expected 
    in your answer when specifying functions and reasoning about their properties:} 
    
    We give a (well-defined) function with domain $R$ and codomain $B$ that is neither one-to-one nor onto.

    Define $g: R \to B$ by, for $(x_1, x_2,x_3) \in R$,
    \[
        g(~(x_1, x_2, x_3)~) = \begin{cases} 
            \A \qquad &\text{if $x_1 = 1$} \\
            \C \qquad &\text{if $x_1 = 0$} \\
            \G \qquad &\text{if $x_1 = -1$}
        \end{cases}
    \]
    This function is well-defined because each ratings $3$-tuples is mapped to a unique base.
    However, this function is not one-to-one, as we can see from the counterexample: 
    $a = (1,1,1)$, $b = (1, 0,0)$. These are ratings $3$-tuples (in the domain) which are 
    distinct (they disagree about the second and third movies) but
    \[
        g(a) = g(~(1,1,1)~) = \A = g(~(1,0,0)~) = g(b)
    \]
    because the two ratings agree on the first movie. Distinct domain 
    elements getting mapped to the same codomain elements is a counterexample to injectivity.

    The function $g$ is also not onto, as we can see from the counterexample $\U$. This is an 
    element of the codomain which is not $f(x)$ for any $x$ in the domain, as we can see 
    from the piecewise definition of $g$, where in no case do we have the output value $\U$.
    \rule{0.5\textwidth}{.4pt}


\end{enumerate}


    
\newpage

\title{HW7 Function and Relations}
\date{Due: Tuesday, November 30, 2021 at 11:00PM on Gradescope}


\maketitle
\thispagestyle{fancy}

{\bf In this assignment,}

You will practice determining and justifying whether 
statements are true in multiple contexts.

Instructions and academic integrity reminders for all homework assignments in 
CSE20 this quarter are on the class website and on the hw1-definitions-and-notations
assignment.

You will submit this assignment via Gradescope
(\href{https://www.gradescope.com}{https://www.gradescope.com}) 
in the assignment called ``hw7-functions-and-relations''.

{\bf Resources}: To review the topics you are working with 
for this assignment, see the class material from Weeks 8 and 9.
We will post frequently asked questions and our answers to them in a 
pinned Piazza post.


In your proofs and disproofs of statements below, justify each  step
by reference to  a component of the  following proof  strategies
we  have discussed so far, and/or to relevant definitions and calculations.
\begin{itemize}
    \item A counterexample can be used to prove that  $\forall x P(x)$ is {\bf false}.
    \item  A witness can be used to prove that  $\exists x P(x)$ is {\bf true}.
    \item {\bf Proof of universal by exhaustion}: To prove that $\forall x \, P(x)$
is true when $P$ has a finite domain, evaluate the predicate at {\bf each} domain element to confirm that it is always T.
    \item  {\bf Proof by universal generalization}: To prove that $\forall x \, P(x)$
is true, we can take an arbitrary element $e$ from the domain and show that $P(e)$ is true, without making any assumptions 
about $e$ other than that it comes from the domain.
    \item To  prove  that $\exists x P(x)$ is {\bf false}, write the universal statement that is 
    logically equivalent to its negation and then prove it true using universal generalization.
    \item {\bf Strategies for conjunction}: To prove that $p \land q$ is true, have two subgoals: 
    subgoal (1) prove $p$ 
is  true; and, subgoal (2) prove $q$ is true. To prove that $p \land q$ is false, it's enough to prove that $p$ is false.
 To prove that $p \land q$ is false, it's enough to prove that $q$ is false.
    \item {\bf Proof of Conditional by Direct Proof}: To prove that the implication $p \to q$ is true, 
    we can assume $p$ is true and use that assumption to show $q$ is true.
    \item {\bf Proof of Conditional by Contrapositive Proof}: To prove that the implication $p \to q$ is true, 
    we can assume $\neg q$ is true and use that assumption to show $\neg p$ is true.
    \item {\bf Proof of disjuction using equivalent conditional}: To prove that the 
    disjunction $p \lor q$ is true, we can rewrite it equivalently as $\lnot p \to q$ and
    then use direct proof or contrapositive proof.
    \item {\bf Proof by Cases}: To prove $q$ when we know $p_1 \lor p_2$, show that $p_1 \to q$ and $p_2 \to q$.
    \item
    {\bf Proof by Structural Induction}: To prove that $\forall x \in X \, P(x)$ where $X$ is a recursively defined set, prove two cases:
        
        \begin{tabularx}{\textwidth}{l X}
        Basis Step: & Show the statement holds for elements specified in the basis step of the definition. \\
        Recursive Step: & Show that if the statement is true for each of the elements used to construct
    new elements in the recursive step of the definition, the result holds for these new elements.
    \end{tabularx}
    
    \item {\bf Proof by Mathematical Induction}: To prove a universal quantification over the set of  all integers greater than  or  equal to some base integer $b$:
    
    \begin{tabularx}{\textwidth}{l X}
        Basis Step: & Show the statement holds for $b$. \\
        Recursive Step: & Consider an arbitrary integer $n$ greater than or  equal to  $b$, assume
        (as the {\bf induction hypothesis})  that the property holds  for $n$, and use  this and
        other facts to  prove that  the property holds for $n+1$.
    \end{tabularx}
    
    \item {\bf Proof by Strong Induction} To prove that a universal quantification over the set of all integers greater than or equal to some  base integer $b$ holds,  pick a  fixed nonnegative integer  $j$ and then: \hfill 
    
    \begin{tabularx}{\textwidth}{l X}
        Basis Step: & Show the statement holds for $b$, $b+1$, \ldots, $b+j$. \\
        Recursive Step: & Consider an arbitrary integer $n$ greater than or  equal to  $b+j$, assume
        (as the {\bf strong  induction hypothesis})  that the property holds  for {\bf each of} $b$, $b+1$, \ldots, $n$, 	
        and use  this and
        other facts to  prove that  the property holds for $n+1$.
    \end{tabularx}

    \item {\bf Proof by Contradiction} 

    To prove that a statement $p$ is true, pick another statement $r$ and once we show
    that $\neg p  \to (r \wedge  \neg r)$ then  we can conclude that  $p$ is  true.
    
    {\it Informally} The statement we care about can't possibly be false, so it must be true.
\end{itemize}

\newpage
{\bf Assigned questions}

\begin{enumerate}
    \item Consider the set $U = \mathcal{P}(\mathbb{R})$. 
    
    \begin{enumerate}
        \item ({\it Translation graded for fair effort completeness; Counterexample graded 
        for correctness})  Translate the statement to English and then give a counterexample 
        that could be used to disprove the statement. You do not need to justify your answer.  
        However, if you include clear explanations, 
        we may be able to give partial credit for an answer with some imprecision.

        {\it Note}: your counterexample should specify a value for $A$ and a value for $B$.
        \[
            \forall A \in U ~\forall B \in U~(~( A \subseteq B \to \lnot (~|A| \geq |B|~)~)
        \]
        \item ({\it Translation graded for fair effort completeness; Counterexample graded 
        for correctness})  Translate the statement to English and then give a counterexample 
        that could be used to disprove the statement. You do not need to justify your answer.  
        However, if you include clear explanations, 
        we may be able to give partial credit for an answer with some imprecision.

        {\it Note}: your counterexample should specify a value for $X$ and a value for $Y$.
        \[
            \forall X \in U~\forall Y \in U~( ~X \subseteq \mathbb{Z}~\land~ Y \subseteq \mathbb{Z} ~\to~  |X|  = |Y| ~)
        \]
        \item ({\it Translation graded for fair effort completeness; Witness graded 
        for correctness})  Translate the statement to English and then give a witness 
        that could be used to prove the statement. You do not need to justify your answer.  
        However, if you include clear explanations, 
        we may be able to give partial credit for an answer with some imprecision.

        {\it Note}: your witness should specify a value for $X$ and a value for $Y$.
        \[
            \exists X \in U~ \exists Y \in U (~(~\mathbb{Z} \subseteq X~) \land
            (~\mathbb{Z} \subseteq Y~) \land \neg ( |X| = |Y|)~)
        \]
    \end{enumerate}

    \item ({\it Graded for correctness}\footnote{This means your solution will be
    evaluated not only on the correctness of your answers, but on your ability to 
    present your ideas clearly and logically. You should explain how you arrived at 
    your conclusions, using 
    mathematically sound reasoning. Whether you use formal proof techniques or 
    write a more informal argument for why 
    something is true, your answers should always be well-supported. Your goal 
    should be to convince the reader that 
    your results and methods are sound.})
    The diagonalization argument constructs, for each function 
    $f: \mathbb{N} \to \mathcal{P}(\mathbb{N})$, a set $D_f$ defined as
    \[
    D_f = \{ x \in \mathbb{N} ~|~ x \notin f(x) \}
    \]
    \begin{enumerate}
        \item Define a function $g$ such that $D_g$ is a finite nonempty set, 
        or explain why no such function exists.
        \item Define a function $h$ such that $D_h$ is an infinite set that is a proper subset of 
        $\mathbb{N}$, or explain why no such function exists.
        \item Define a function $k$ such that $D_k$ is a proper superset
        of $\mathbb{N}$ (in other words, $\mathbb{N}$ is a proper subset 
        of $D_k$), or explain why no such function exists.\footnote{
        For sets $A$ and $B$, when the relation $A\subseteq B$ holds we say that $A$ 
        is a subset of $B$ and that $B$ is a superset of $A$. Similarly, when the relation 
        $A\subsetneq B$ holds we say that $A$ is a proper subset of $B$ and that $B$ is a 
        proper superset of $A$. 
        }
    \end{enumerate}    
    
    \item ({\it Graded for correctness}) For each part of this question, you do not need to justify your answer.  
    However, if you include clear explanations, 
    we may be able to give partial credit for an answer with some imprecision.
        
    \begin{enumerate}
    \item Recall that 
    in a movie recommendation system, each 
    user's ratings of movies is represented as a $n$-tuple (with the positive integer $n$ 
    being the number of movies in the database), and each component of 
    the $n$-tuple is an element of the collection $\{-1,0,1\}$. Assume there are five movies in the database, 
    so that each user's ratings
    can be represented as a $5$-tuple. We call $Rt_5$ the set of all ratings $5$-tuples.
    Consider the binary relation on the  set of all 
    $5$-tuples where each  component of the $5$-tuple is an element of the collection $\{-1,0,1\}$:
    \[
        G = \{ (u,v) \in Rt_5 \times Rt_5 \mid \text{the number of $0$s in $u$ is the same as the number of $0$s in $v$} \}
    \]
    This is an equivalence relation (you do not need to prove this).

    Recall that the {\bf equivalence class} of an element $x \in X$ for an equivalence relation $\sim$ on the set $X$ 
    is the set $\{s \in X | (x, s) \in \sim \}$. We write this as $[x]_\sim$.
    
    \begin{enumerate}
        \item Find a ratings $5$-tuple $v$ such that $[v]_{G} = \{v \}$.
        \item Find distinct ratings $5$-tuples $u_1, u_2$ ($u_1 \neq u_2$) whose equivalence classes $[u_1]_{G}$ and $[u_2]_{G}$ have the same size.
        \item Find distinct ratings $5$-tuples $w_1, w_2$ ($w_1 \neq w_2$) whose equivalence classes $[w_1]_{G}$ and $[w_2]_{G}$ have different sizes.
    \end{enumerate}

    \item Let $S_{1,2}$ be the set of RNA strands of length $1$ or $2$, formally 
        \[
            S_{1,2} = \{ s \in S \mid ( rnalen(s) = 1) \lor (rnalen(s) = 2)\}
        \]
    Consider the binary relation on $S_{1,2}$ given by
    \begin{align*}
        P = \{ (s, s') \in S_{1,2} \times S_{1,2} \mid &\text{$s$ is a prefix of $s'$, }\\
        &\text{namely either $s = s'$ or there is 
        some base $b$ such that $sb=s'$} \}
    \end{align*}
    This is a partial ordering (you do not need to prove this).

    Draw the Hasse diagram of $P$.

    \item Consider the set $CP$ of compound propositions that use propositional variables from the set $\{p,q\}$.
    We define the logical equivalence binary relation on this set by 
    \[
        LE = \{ (x, y) \in CP \times CP \mid x \equiv y \}
    \]
    This is an equivalence relation (you do not need to prove this).

    \begin{enumerate}
        \item Give two distinct examples of elements in $[~(p \land \lnot p)~]_{LE}$
        \item Give two distinct examples of elements in $[~(p \to q)~]_{LE}$
    \end{enumerate}

    {\it Bonus - not for credit; do not hand in}: 
    Prove that $G$ is an equivalence relation on
    the set of ratings $5$-tuples.
    Prove that $P$ is a partial ordering on $S_{1,2}$.
    Prove that $LE$ is an equivalence relation on 
    the set of compound propostions that use propositional variables from 
    the set $\{p,q\}$.



    \end{enumerate}

    \item Imagine you are playing the role of Alice in the Diffie Hellman key agreement (exchange) protocol.  
    You and Bob have agreed to use the prime $p = 7$ and its primitive root $a = 3$.
    Your secret integer is $k_1 = 3$.
        
        \begin{enumerate}
        \item  ({\it Graded for fair effort completeness}\footnote{This means 
        you will get full credit so long as your submission demonstrates honest 
        effort to answer the question. You will not be penalized for incorrect answers.}) 
        Calculate the number you send to Bob, 
        $a^{k_1} \textrm{\bf ~mod~} p$.  Use the modular exponentiation algorithm
        for the calculation. Include a trace of the algorithm in your solution.

        \input{../activity-snippets/modular-exponentiation-algorithm.tex}
        
        \item  ({\it Graded for fair effort completeness}) Bob sends you the number $5$. Compute  your shared key, $\left(a^{k_2}\right)^{k_1}  
        \textrm{\bf ~mod~} p$.
        Hint: $a^{k_2} \textrm{\bf ~mod~} p$ is what Bob sent you.  Include all relevant calculations, annotated with explanations, 
        for full credit.
        
        
        \item ({\it Graded for fair effort completeness}) What are some possible values for Bob's secret integer?  What 
        algorithm are you using to compute them?
        \end{enumerate}
    

\end{enumerate}


    
\newpage

\title{Project}
\date{Part 1 due 10/14/21; Part 2 due 11/4/21; Part 3 due 12/2/21}


\maketitle
\thispagestyle{fancy}
The project component of this class will be an opportunity for you to extend your 
work on assignments and explore applications of your choosing. 

{\it Why?}
To go deeper and explore the material from discrete math and how it relates to Computer Science.

{\it How?} During emergency remote instruction last academic year, we discovered
that video assessement and some open-ended personalized projects help ensure fairness
and can be less stressful for students than in-person midterm exams. Asynchronous project
submission also gives flexibility and allows more physical distancing.

Your videos: We will delete all the videos we receive from you after assigning final grades for the course, 
and they will be stored in a university-controlled Google Drive directory 
only accessible to the course staff during the quarter. 
Please send an email to the instructor (minnes@eng.ucsd.edu) if you have 
concerns about 
the video / screencast components of this project or cannot complete projects in this style for some reason.

You may produce screencasts with any software you choose. 
One option is to record yourself with Zoom; a tutorial on how to use Zoom to record a 
screencast (courtesy of Prof. Joe Politz)  is here: 

\url{https://drive.google.com/open?id=1KROMAQuTCk40zwrEFotlYSJJQdcG_GUU}.

The video that was produced from that recording session in Zoom is here:

\url{https://drive.google.com/open?id=1MxJN6CQcXqIbOekDYMxjh7mTt1TyRVMl}

\subsection*{What resources can you use?}
This project must be completed individually, without any help from other people, 
including the course staff (other than logistics support if you get stuck with screencast). 

You can use any of this quarter's CSE 20 offering (notes, readings, class videos, homework feedback). 
These resources should be more than enough. If you are struggling to get started and want to 
look elsewhere online, you must acknowledge this by listing and citing any resources you consult 
(even if you do not explicitly quote them). Link directly to them and include the name of the 
author / video creator and the reason you consulted this reference. The work you submit for 
the project needs to be your own. Again, you shouldn't need to look anywhere other 
than this quarter's material and doing so may result in definitions or notations 
that conflict with our norms in this class so think carefully before you go down this path.

The project has three parts. 
\begin{itemize}
    \item Part 1 of Project: due Thursday October 14
    \item Part 2 of Project: due Thursday November 4
    \item Part 3 of Project: due Thursday December 2
\end{itemize}

\newpage
\subsection*{Part 1: due Thursday October 14}
\subsubsection*{Written component}
\begin{enumerate}
\item 

In Week 1, we discussed the mathematical definition of a function, namely
that a {\bf function} is defined by its (1) domain, (2) codomain, and (3) rule assigning each 
element in the domain exactly one element in the codomain.

\begin{enumerate}
    \item Write out the definition for an example function you make up.
    You can choose any example you like, so long as it is your own independent 
    effort and it is not a function from a class example, homework, or Review Quiz in this class so far.
    Use the notation defined in class. Label and define the domain, codomain, and rule clearly.
    \item Calculate the result of applying your function to an element in its domain.
    Just like in homework, include (clear, correct, complete) calculations and/or references to definitions
    to present your function application.
\end{enumerate}

\item In CSE 20 this quarter, we will be exploring the applications of discrete mathematics for core CS topics. 
The following videos introduce some of these topics and the 
work happening here at UCSD to explore them. Pick one of the following videos, watch it, and 
then write a few sentences answering the following:
\begin{enumerate}
\item Which video did you watch? Why did you choose the video you watched?
\item What followup question(s) would you like to ask the person in the video about their work?
At least one of your followup questions should be about a technical aspect of the work that you
would like to learn more about.
\end{enumerate}

({\it Click on the video titles below for the links})

\href{https://www.youtube.com/watch?v=PrAoks7OhE8}{Bioinformatics and virology: 
{\it Niema Moshiri - Genome Sequence Alignment}}

\href{https://www.uctv.tv/computer-science/search-details.aspx?showID=33425}{Human robotics interaction: 
{\it Angelique Taylor - Improving Human-Robot Interaction}}

\href{https://www.uctv.tv/computer-science/search-details.aspx?showID=33421}{Computer vision: 
{\it Manmohan Chandraker: Giving Computers the Gift of Vision}}

\href{https://www.uctv.tv/computer-science/search-details.aspx?showID=33423}{Data centers and energy efficiency:
{\it  Max Mellette: Improving Data Centers with Photonics}}

\href{https://www.uctv.tv/computer-science/search-details.aspx?showID=33420}{Programming languages 
and data structures: {\it Nadia Polikarpova: Creating New Languages for Programming}}

\href{https://www.uctv.tv/computer-science/search-details.aspx?showID=34350}{Machine learning (and surfing) 
for climate science: {\it Studying Climate Change Through Surfing with Smartfin - We Are CSE: Jasmine Simmons}}
\end{enumerate}
\subsubsection*{Video component}
Presenting your reasoning and demonstrating it via screenshare are important 
skills that also show us a lot of your learning. Getting practice with this style of 
presentation is a good thing for you to learn in general and a rich way for us to assess your skills. 

Prepare a 3-5 minute screencast video that starts with 
your face and your student ID for a few seconds at the beginning, and introduce yourself audibly while on screen. 
You don't have to be on camera for the rest of the video, though it's fine if you are. 
We are looking for a brief confirmation that it's you creating the video and doing the work 
submitted for the project.

Then, explain your work in question 1 of the written component.
Discuss at least one potential mistake that someone solving 
a similar question should avoid (this could be a mistake you made while thinking about this 
problem or something you anticipate a classmate might struggle with); explain why the 
mistake is wrong and how to fix it.

Finally, explain any differences between your pre-survey description 
of ``what a function is" and the mathematical definition of what a function is. 
You should have an email copy of your responses to the pre-survey, 
and you can refer to what you wrote in your explanation.

Use this Google form

\url{https://forms.gle/SLd8SrJdXR5HCLQr7}  (click to follow link) 

to directly upload a video file for this assignment.
It should be a file that you can easily play on your system. 
One way you can determine if this is true is if you can store it on your Google Drive and play it from there,
since that's how we will watch it.

\subsubsection*{Checklist (this is how we will grade Part 1 of the project)}
\begin{itemize}
\item Question 1
    \begin{itemize}
        \item The function definition is complete and uses correct notation, 
        and is different from class, homework, and quiz examples.
        \item The calculation of the function application is correct and is 
        supported by clear, correct, complete justification.
    \end{itemize}
\item Question 2
    \begin{itemize}
        \item (At least) one of the videos is mentioned and a reason for selecting it is included.
        \item At least one technical question in described that is connected to the video selected.
    \end{itemize}
\item Video
    \begin{itemize}
        \item Video loads correctly and is between 3 and 5 minutes. It includes your face and your student ID, 
        and you introduce yourself audibly while on screen.
        \item Video presents your solution for Question 1.
        \item A potential mistake is presented and discussed.
        \item The mathematical definition of function is compared to your response in the pre-survey.
    \end{itemize}
\end{itemize}

\newpage
\subsection*{Part 2: due Thursday November 4}
\subsubsection*{Written component}
\begin{enumerate}
\item In this part of the project, you will select one question from one of the review quizzes 
from 10/4/2021 (Monday of Week 1) to 10/29/2021 (Friday of Week 5) to revisit. 
Include the problem statement, why you picked this question (e.g. what is interesting about it, 
what is hard about it, or why you wanted to take a second look at it), and your solution. 
    \begin{itemize}
        \item Question selection: you can pick any {\bf one question} listed in the Review 
        sections of the relevant notes documents, and you must address all of its parts.
        \item For each part of your chosen question: prepare a complete solution 
        (you can use the homework solutions we post for guidance about the style). 
        Your submission will be evaluated not only on the correctness of your answers, 
        but on your ability to present your ideas clearly and logically. 
        You should explain how you arrived at your conclusions, using mathematically 
        sound reasoning. Your goal should be to convince the reader that your results 
        and methods are sound. Imagine you are preparing these solutions for someone else 
        taking CSE 20 who missed that week and is ``catching up".
    \end{itemize}
\item In this part of the project, you'll consider the importance of data types
and precision in Computer Science. 
Read three articles

(1) This discussion of the causes of a wide-spread problem in published genomics papers

\url{https://www.nature.com/articles/d41586-021-02211-4} (Click to follow link)

from the journal Nature; 

(2) this IEEE profile of Katherine Johnson, 

{\tiny \url{https://spectrum.ieee.org/the-institute/ieee-history/katherine-johnson-the-hidden-figures-mathematician-who-got-astronaut-john-glenn-into-space}}

(Click to follow link) a NASA ``computer" who calculated trajectories for 
early space exploration and who passed away in 2020; and 

(3) this NASA report about the unsuccessful 1999 Mars Climate Orbiter mission

\url{https://solarsystem.nasa.gov/missions/mars-climate-orbiter/in-depth/}

In one or two sentences, summarize the main lesson you draw from each article.

Thinking back to your own experiences, give an example of when you used computers or Computer Science
to help you *avoid* an error. Also, give an example when your use of computers or Computer Science
*caused* an error.

What measures do you take to increase your confidence in the results of your own human and digital 
(i.e. machine) computation? Why do you think these are sufficient?


\end{enumerate}

\subsubsection*{Video component}

Presenting your reasoning and demonstrating it via screenshare are important skills that 
also show us a lot of your learning. Getting practice with this style of presentation 
is a good thing for you to learn in general and a rich way for us to assess your skills. 

Prepare a 3-5 minute screencast video explaining your work in question 1 of the written component.
During your solution presentation, point out at least one potential mistake that someone 
solving a similar question should avoid (this could be a mistake you made while thinking 
about this problem or something you anticipate a classmate might struggle with); 
explain why the mistake is wrong and how to fix it. 

You do not need to include complete details of every part of your solution. 
It is up to you to choose what is most important so that you can stick to the 
timing guidelines and still have time to include discussing potential mistakes.

Include your face and your student ID (we'd like a photo ID that includes your name 
and picture if possible) for a few seconds at the beginning, and introduce yourself 
audibly while on screen. You don't have to be on camera the whole time, though it's fine 
if you are. We are looking for a brief confirmation that it's you creating the 
video/doing the work attached to the video.

Use this Google form

\url{https://forms.gle/SLd8SrJdXR5HCLQr7}  (click to follow link) 

to directly upload a video file for this assignment.
It should be a file that you can easily play on your system. 
One way you can determine if this is true is if you can store it on your Google Drive and play it from there,
since that's how we will watch it.

\subsubsection*{Checklist (this is how we will grade Part 2 of the project)}
\begin{itemize}
\item Question 1
    \begin{itemize}
        \item Selected review quiz question is labelled clearly, including the day 
        it belongs to and the statement of the question.
        \item Solution is complete: it addresses each part of the review quiz question selected.
        \item Solution is correct: it clearly and correctly justifies the correct answer 
        for each part of the question. You are welcome to check your answers with the 
        Gradescope autograder (we will be reopening the review quizzes for this purpose). 
        We will evaluate your submissions for the quality of your justification.
    \end{itemize}
\item Question 2
    \begin{itemize}
        \item A key lesson from each of the three references is stated clearly and 
        is relevant to the message of the articles. Supporting explanations are included.
        \item A specific example of an instance where using computers/ CS *caused* an error is described.
        \item A specific example of an instance where using computers/ CS helped *avoid* an error is described.
        \item Lesson(s) are drawn from the previous experiences.
        \item Specific strategies for increasing confidence in computation are described and justified.
    \end{itemize}
    \item Video
    \begin{itemize}
        \item Video loads correctly and is between 3 and 5 minutes. It includes your face and your student ID, 
        and you introduce yourself audibly while on screen.
        \item Video presents your solution for Question 1.
        \item A potential mistake is presented and discussed.
    \end{itemize}
\end{itemize}

\newpage
\subsection*{Part 3: due Thursday December 2}
\subsubsection*{Written component}
\begin{enumerate}
    \item In this part of the project, you will analyze a quantified statement about RNA strands. 
    The definitions for RNA strands are available in the class notes. 
    Example quantified statements about RNA strands are in the homework. 
    Complete the following:
    \begin{enumerate}
        \item Write a quantified statement symbolically. Your quantified statement should satisfy {\bf all} of the following requirements:
            \begin{itemize}
                \item Have a nesting of quantifiers with at least one forall quantification 
                and at least one existential quantification.
                \item Have at least one negation and at least one binary logic operation (and, or, xor, if, iff).
                \item Negations appear only within predicates (that is, so that no negation is outside 
                a quantifier or an expression involving logical connectives).
                \item Use {\bf exactly one} of the predicates Mut, Ins, Del.
                \item Not be a statement we have analyzed already in class materials.
            \end{itemize}
        \item Translate your statement from part a. to English.
        \item Negate the whole statement from part a. and rewrite this negated statement so 
        that negations appear only within predicates (that is, so that no negation is outside a 
        quantifier or an expression involving logical connectives).
        \item Prove or disprove your statement from part a.
    \end{enumerate}
    \item In this part of the project, you'll consider the impact Computer Science has on society.
    Read two articles:

    (1) This policy piece about facial recognition software

    {\tiny \url{https://thehill.com/policy/technology/569543-federal-agencies-planning-to-expand-use-of-facial-recognition}}
    (Click to follow link); and 

    (2) this exploration of accessibility for visually impaired website users
    
    {\tiny \url{https://www.wsj.com/articles/colorblind-users-push-technology-designers-to-use-signals-beyond-color-11591351201}}
    (Click to follow link).

    Update (November 18): the Wall Street Journal about web design is now behind a pay wall. If you 
    do not have access to it, please read the following two articles instead:

    {\tiny \url{https://webaim.org/articles/visual/colorblind}}
    (Click to follow link) and
    {\tiny \url{https://webaim.org/resources/designers/}}
    (Click to follow link)


    In one or two sentences, summarize the main lesson you draw from each article.

    Give an example of an algorithm or computer system that you use or have been used by others to 
    make some decision that affects you. The algorithms may be institutional or personal, formal or 
    heuristic, and should output a 
    specific result or decision. Explain your answer, either by reference to your knowledge of 
    the algorithm itself or by observations you make 
    about outputs of the algorithm.

    Give an example of how someone different from you might have a different experience 
    with this algorithm. Support your example with lessons you learned in the 
    readings, citing what you learned from the articles.
\end{enumerate}

\subsubsection*{Video component}
Presenting your reasoning and demonstrating it via screenshare are important skills that 
also show us a lot of your learning. Getting practice with this style of presentation 
is a good thing for you to learn in general and a rich way for us to assess your skills. 

Prepare a 3-5 minute screencast video explaining your work in question 1 parts (c) and (d)
of the written component (i.e. the negation and proof).
During your solution presentation, point out at least one potential mistake that someone 
solving a similar question should avoid (this could be a mistake you made while thinking 
about this problem or something you anticipate a classmate might struggle with); 
explain why the mistake is wrong and how to fix it. 

You do not need to include complete details of every part of your solution to these parts. 
It is up to you to choose what is most important so that you can stick to the 
timing guidelines and still have time to include discussing potential mistakes.

Include your face and your student ID (we'd like a photo ID that includes your name 
and picture if possible) for a few seconds at the beginning, and introduce yourself 
audibly while on screen. You don't have to be on camera the whole time, though it's fine 
if you are. We are looking for a brief confirmation that it's you creating the 
video/doing the work attached to the video.

Use this Google form

\url{https://forms.gle/SLd8SrJdXR5HCLQr7}  (click to follow link) 

to directly upload a video file for this assignment.
It should be a file that you can easily play on your system. 
One way you can determine if this is true is if you can store it on your Google Drive and play it from there,
since that's how we will watch it.

\subsubsection*{Checklist (this is how we will grade Part 3 of the project)}
\begin{itemize}
    \item Question 1
        \begin{itemize}
            \item Quantified statement is clearly stated, is well-defined, syntactically correct, and meets all requirements.
            \item Translation to English is clear, correct, and complete.
            \item The negation of the quantified statement is clearly stated, is well-defined, syntactically correct, and meets all requirements.
            \item The proof or disproof of the original statement is clear, correct, and complete. 
        \end{itemize}
    \item Question 2
        \begin{itemize}
            \item A key lesson from each of the two references is stated clearly and 
            is relevant to the message of the articles. Supporting explanations are included.
            \item A specific example of an algorithm or computer systems that impacts you is described.
            \item A description of how this algorithm or computer system might impact someone different from you
            differently is included, and is supported with references to one or both of the articles.
            \end{itemize}
\item Video
    \begin{itemize}
        \item Video loads correctly and is between 3 and 5 minutes. It includes your face and your student ID, 
        and you introduce yourself audibly while on screen.
        \item Video presents your solution for Question 1 parts (c) and (d).
        \item A potential mistake is presented and discussed.
    \end{itemize}
\end{itemize}
\newpage

\end{document}